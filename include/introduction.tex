\section{Введение}

\qquad Теорией игр называется математическая теория принятия решений
в конфликтных ситуациях. В простейших моделях рассматривается \textit{лицо
принимающее решение} (ЛПР), которое выбирает своё действие из некоторого
множества стратегий. Считается, что задана \textit{целевая функция},
которая отражает интересы ЛПР и зависит от выбранной им стратегий.
Задача принятия решений состоит в том, чтобы найти стратегию, 
при которой достигается максимум целевой функции. Отличие конфликтной ситуации
заключается в том, что решения принимаются не одним лицом, 
а всеми участниками конфликтной ситуации и функция выигрыша
каждого индивида зависит не только от его решения, но 
и от решения остальных участников.  Модель такого вида называется --
игрой, а участники конфликта -- игроками. В рамках данной работы
будет рассмотрена задача из теории \textit{некооперативных игр} -- 
игр, в которых игроки действуют самостоятельно, независимо друг 
от друга. 

	Сложность задачи многокритериальных игр состоит в объёмы вычислений
необходимых для поиска точного результата. Методы предложенные \cite{shapley},
\cite{novikova} сокращают вычислительную сложность предлагая 
достаточно точные результаты. Одним из методов решения конечных 
многокритериальных игр является метод свёрток \cite{blackwell}.
Он позволяет скаляризовывать критерии, заменяя исходную задачу на 
семейство более простых задач. 

	Существующие методы решения этого типа предполагают использование всеми игроками одной функции свёртки (линейной свёртки \cite{shapley} или обратной логической свёртки \cite{novikova}). Однако, можно предположить,
что игроки не сговариваются перед началом игры, не обсуждают использование 
свёрток, на основе которых они будут выбирать стратегию, следовательно, можно
предположить, что они могут использовать разные свёртки. Именно такой подход
будет анализироваться в текущей работе.