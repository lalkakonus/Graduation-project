\section{Введение}

Теорией игр называется математическая теория принятия решений
в конфликтных ситуациях. В простейших моделях рассматривается \textit{лицо
принимающее решение} (ЛПР), которое выбирает своё действие из некторого
множества стратегий. Считается, что задана целевая функция,
которая отражает интресы ЛПР и зависит от выбранных им стратегий.
Задача принятия решений состоит в том, чтобы найти стратегию, 
доставляющую максимум целевой функции. Отличие конфликтной ситуации
заключается в том, что решения принимаются не одним лицом, 
а всеми участниками конфликтной ситуации и функция выигрыша
кадого индивида зависит не только от его решения, но 
и от решения остальных участников.  Модель такого вида называется -
игрой, а учаастники конфликта - игроками. В рамках данной работы
будет рассмотрена задача из теории \textit{некооперативных игр} - 
игр, в которых игроки действуют самостоятельно, независимо друг 
от друга. Определим формально модель игры с несколькими участниками в общем виде.

Есть конечное \textit{множество игроков} $A$, которые перенумерованы
$1, 2, ..., m$. Каждый игрок $a \in A$ имеет 
\textit{множество чистых стратегий} $S_a=\{1,2,...,n_a\}$, при этом 
\textit{игровой ситуацией} или просто \textit{ситуацией}
называется $m$-мерный вектор:

\begin{equation}
	\textbf{s} = (s_1, \, s_2, \ldots , \, s_m) \in
	\Motimes_{a \in A} S_a
	\label{eq:game_situation}
\end{equation}

\textit{Функция выигрыша} обозначает выигрыша игрока при конкретной ситуации в игре. Она определена для каждого игрока из $A$ и имеет вид:

\begin{equation}
	F: S_1 \times S_2 \times ... \times S_m \rightarrow \mathbb R
	\label{eq:single_dim_payoff_function}
\end{equation}

\newtheorem{Def}{Определение}
\begin{Def}
	Игрой в нормалной форме называется совокупность:
	\begin{equation}
		G = \big \langle A, S, F \big \rangle
		\label{eq:normal_form_game}
	\end{equation}
	где: 
 
	$ A = \{1, \, 2, ..., \, m\}$ - множество игроков,

	$ S = \{S_1, \, S_2, ..., \, S_m\}$ - множество наборов чистых
	стратегий игроков,

	$ F = \{F_1, \, F_2, ..., \, F_m\}$ - множество функций выигрыша
	игроков.
\end{Def}

Теперь введём фундаментальное понятие в теории игр - 
\textit{равновесие по Нэшу}:

\begin{Def}
	Ситуация $\textbf{s} = (s_1^0, \, s_2^0, ..., \, s_m^0)$ называется
	равновесием по Нэшу игры 
	$G = \big \langle A, S, F \big \rangle$, если:
	\begin{equation}
		\max \limits_{s_a \in S_a} 
		F_a(s_1^0, ..., s_{a-1}^0, s_a, s_{a+1}^0..., s_m^0)=
		F_a(s_1^0, ..., s_{a-1}^0, s_a^0, s_{a+1}^0..., s_m^0),
		\: a \in A	
		\label{eq:nash_equilibrium}
	\end{equation}
\end{Def}

Смысл этого определения заключается в том, что 
при ситуации в игре, которая является равновесием по Нэшу,
одному игроку индивидуально не выгодно отклоняться от своей стратегии.

До этого мы рассматривали функции выигрыша 
игроков, которые имели вид:
\eqref{eq:single_dim_payoff_function}, т.е. каждому игроку 
соответсвовало одно значение, зависящее ситуации игры.
Однако не всегда интересы могут быть выражены одним критерием. Часто 
возникают разные оценки качества принимаемого решения, причем они могут 
быть противоречивыми и их нельзя свести друг у другу. Например 
характеристиками решени могут быть значения \textit{(время, деньги)} или 
\textit{(математическое ожидание, дисперсия)}. Следуя этим рассуждениям  
рассмотрим обобщение игры \eqref{eq:normal_form_game} такое, что функция
выигрыша игроков имеет вид:

\begin{equation}
	F: S_1 \times S_2 \times ... \times S_m \rightarrow \mathbb R^m
	, \; m \in \mathbb{N}
	\label{eq:multidem_payoff_function}
\end{equation}

Такое обощение ближе к реальным ситуациям
в которых рассматриваются несколько значимых параметров. Для примера 
такой игры можно привести задачу выбора машины: допустим покупателю важно
чтобы машина имела большую мощность, достаточный уровень безопасности 
и мало стоила, продавцу же важно, чтобы она стоила как можно дороже и 
кроме того следует продавать машины из которые плохо покупают. 
Таким образом мы получили игру, в которой игроки имеют два и тра критерия
соответсвенно, которые важны для них при выборе стратегии. 
Пока что мы допустили существование игры с такой
функцие выигры, формализцию и подробное описание будет позже.

Приведённые выше обощения приводят нас к другому разделу
математики, а именно -- \textit{многокритериальной оптимизации}.
Рассмотрим следующую задачу которая относится к этой области.

\begin{equation}
	\max\limits_{x \in это симплекс X} F(x)=({f}_1(x),\ldots, {f}_n(x))
	, \; X \subseteq \mathbb{R}^n	
	\label{eq:multi_criteral_problem}
\end{equation}

Это задача заключается в том, что у нас есть $n$-мерная функция,
которая представляет собой множество значений критериев, зависящая
от параметров, которые принадлежат некоторому множеству.
Особенность заключается в том, что правило сравнения двух векторов
не определено однозначно, т.е. в общей задаче не всегда можно точно сказать,
какой из двух векторов значений функции предпочтительнее.
Для внесения определённости в задачу, введём понятие
\textit{оптимальных по Парето} и \textit{оптимальных по Слейтеру}
векторов в задаче \eqref{eq:multi_criteral_problem}.

\begin{Def}
	Допустимое решение $\hat{x}\in{X}$ называется 
	эффективным по Слейтеру (эффективным по Парето) для задачи
		
	$$	
		\max\limits_{x \in X} F(x)=({f}_1(x),\ldots, {f}_n(x))
		, \; X \subseteq \mathbb{R}^n	
	$$	
	
	если \textbf{не} существует $x\in{X}$ такого, что 
	$f_k(x)>f_k(\hat{x}) \; (f_k(x) \geqslant f_k(\hat{x}))$ для всех
	$k=\{1,...,n\}$. Множетсво всех эффективных по Слейтеру 
	(эффективных по Парето) решений задачи 
	\eqref{eq:multi_criteral_problem} называется 
	\textit{множеством Слейтера} (\textit{множеством Парето}) 
	задачи \eqref{eq:multi_criteral_problem}.
\end{Def}

Другими словами это такое множество значений, 
при котором значение каждого частного показателя, характеризующего систему, не может быть улучшено без ухудшения других.

Обычно в задача многокритериальной оптимизации требуется (???)
найти множество Слейтера. Для этих целей существуют разные методы,
в работе далее исследуюется \textit{метод свёрток},
изначально предложенный Л.С. Шепли \cite{shapley}. 
Он заключается в том, что задача \eqref{eq:multi_criteral_problem}
заменяется параметрическим семейством скалярных задач 

	$$
		\max\limits_{x \in X} C(\{f_i\}_{i=1}^{n}, \lambda, x), \;
		\lambda \in \Lambda
	$$
	где: 
 
	$C$ – функция свертки частных критериев $\{f_i\}_{i=1}^n$ задачи 
\eqref{eq:multi_criteral_problem} в единый скалярный критерий,
 
	$\lambda$ – параметр свертки заданный на некоторой
области определения $\Lambda$.
\newline

В текущей работе рассмотрены две различные свёртки --
\textit{линейная свёртка} и \textit{обратная логическая свёртка}:

\begin{Def}
	Линейной свёрткой с параметром $\lambda$ для функции критериев задачи
	\eqref{eq:multi_criteral_problem} называется функция:
	\begin{equation}
		L(\{f_i\}_{i=1}^{n}, \lambda, x) = 
		\sum_{i=1}^{n} \lambda_i f_i,
		\label{eq:linear_scalarization}
	\end{equation}
	где  
	\begin{equation}
		\lambda \in 
		\Lambda = \{
			(\lambda_1, \ldots, \lambda_n) \:
			| \: \sum_{i=1}^n \lambda_i = 1, \: 
			  \lambda_i \geq 0 \; i = 1, \ldots n 
		\}.		
		\label{eq:Lambda}
	\end{equation}
\end{Def}

\begin{Def}
	свёрткой Гермейера с параметром $\mu$ для 
	функции критериев задачи \eqref{eq:multi_criteral_problem}
	называется функция:
	
	\begin{equation}
		G(\{f_i\}, \mu, x)=
		\min \limits_{i: \mu_i > 0} \mu_i f_i,
		\label{eq:germeyer_scalarization}	
	\end{equation}
	где 
	\begin{equation}
		\mu \in 
		M = \{
			(\mu_1, \ldots, \mu_n) \:
			| \: \sum_{i=1}^n \mu_i = 1, \: 
			  \mu_i \geq 0 \; i = 1, \ldots n 
		\}.
		\label{eq:Mu}	
	\end{equation}
\end{Def}

Применение линейной свертки в задачах вида
\eqref{eq:multi_criteral_problem} обосновывается теоремой Карлина:

\newtheorem{Th}{Теорема}
\begin{Th}[Карлин \cite{carlin}]
	Рассмотрим задачу \eqref{eq:multi_criteral_problem}. 
	Пусть множество $X \subseteq \mathbb{R}^n$ выпукло,
	а функции $f_1, \ldots, f_n$ - вогнуты на нём.
	Если $x^*$ – эффективная по Парето точка,
    тогда существует вектор $\lambda \in \Lambda$ из 
    \eqref{eq:Lambda} такой, что $x^*$ является точкой 
    максимума функции \eqref{eq:linear_scalarization} по переменной
    $x$.   
\end{Th}

Гермейером была предложена свертка, которая также аппроксимирует 
множество Слейтера. Её применение в многокритериальных 
задачах обосновывается следующей теоремой:

\begin{Th}[Гермейер \cite{germeyer}]
	Рассмотрим задачу \eqref{eq:multi_criteral_problem}.     
    Пусть $x^*$ - эффективная по Слейтеру точка, 
    причем $f_i(x^*)>0, \ldots, f_n(x^*)>0$.
    Тогда существует вектор $\mu \in M$ из \eqref{eq:Mu} 
    и $x^*$ является точкой максимума функции 
    \eqref{eq:germeyer_scalarization} по переменной $x$. 
\end{Th}

В работе будет использоваться модификация свёртки Гермейера -
\textit{обратная логическая свертка}. Она отличается
только тем, что веса стоят в знаменателе, а не в числителе.

\begin{Def}
	Обратной логической свёрткой с параметром $\mu$ для 
	функции критериев задачи \eqref{eq:multi_criteral_problem}
	называется функция:
	
	\begin{equation}
		G(\{f_i\}, \mu, x)=
		\min \limits_{i: \mu_i > 0} \frac{f_i}{\mu_i},
		\label{eq:germeyer_scalarization}	
	\end{equation}
	где
	\begin{equation}
		\mu \in 
		M = \{
			(\mu_1, \ldots, \mu_n) \:
			| \: \sum_{i=1}^n \mu_i = 1, \: 
			  \mu_i \geq 0 \; i = 1, \ldots n 
		\}.
		\label{eq:Mu}	
	\end{equation}
\end{Def}

Теперь вернёмся к рассмотрению некооперативных игр.
Если множество чистых стратегий у игроков конечно, то игра
называется \textit{конечная}. Для конечной игры можно
определить обобщение модели, 
а именно - \textit{смешанное расширение игры}.
Смысл смешанного расширения игры заключается в том
все игроки выбирают каждую из своих чистых стратегий с некоторой
фиксированной то вероятностью, и его стратегией является
не одна чистая стратегия, а вероятностное  распределение над множеством 
его чистых стратегий. Выигрышем игрока в таком случае считаем взвешенный
выигрыш по всем ситуациям с весами соответсвующими вероятностям данной
ситуации. Определим эти понятия формально.

Пусть в игре 
$G = \big \langle A, S, F \big \rangle$. $A$ -- конечное множество игроков,
которые перенумерованы $1, 2, ..., m$, причём
множества стратегии $S_a=\{1, 2,...,n_a\} \; a \in A$ конечны.  
\textit{Смешанной стратегией} игрока $a \in P$ называется 
вероятностоное распределение над множеством чистых стратегий 
$S_a$ игрока $a \in A$:
$$
	\pi^a=(\pi^a_1, \pi^a_2, ..., \pi^a_{n_a}),
$$
где $\pi^a_i$ - это веротяность выбора игрком $a \in A$ чистой 
стратегии $s^a_i$ в качестве реальной стратегии игрока.
Распределение является элементом симлекса:
$$
	P^a = \{ 
		\pi^a=(\pi^a_1, \pi^a_2, ..., \pi^a_{n_a}) \: | \:
		\sum \limits_{i=1}^{n_a} \pi^a_i = 1, \; \pi^a_i \in [0,1] \; 
		i = 1, 2, ..., n_a
		\}
$$
которое называется \textit{множество смешанных стратегий игрока}.
Введём обозначение для заданного набора стратегий:
$$
	\pi = (\pi^1, \ldots, \pi^m) \in P = \Motimes_{a \in A} P^a,
$$
и вероятности реализации ситуации \textbf{s} из 
\eqref{eq:game_situation}:
$$
	p(s|\pi) \myeq \prod_{a \in A} \pi^a_{s_a}
$$
Тогда математическое ожидание выигрыша игрока $a \in A$
задаётся функцией: 
$$
	\overline F_a(\pi) = \sum \limits_{s \in S}p(s|\pi)F_a(s),
$$
где функция $F_a$ - определена в \eqref{eq:normal_form_game}.
Таким образом смешанное расширение игры в нормальной форме определяется
следующим образом. 

\begin{Def}
	Смешанным расширением игры в нормалной форме называется 
	совокупность:
	\begin{equation}
		\overline G = \lbrace A, S, \overline F \rbrace	
		\label{eq:ext_normal_form_game}
	\end{equation}
	где: 
 
	$A = \{1, 2, ..., m\}$ -- множество игроков,

	$P = \Motimes_{a \in A} P^a$ -- множество наборов смешанных 
	стратегий игроков,

	$\overline F = \{
		\overline F_1, \overline F_2, ..., \overline F_m
	\}$
	-- множество функций выигрыша игроков.
\end{Def}

Ситуации равновесия \eqref{eq:nash_equilibrium} игры $\overline G$
будем называть ситуациями равновесия в смешанных стратегиях игры $G$
или смешанными равновесиями по Нэшу. 

Теперь в игре \eqref{eq:ext_normal_form_game} будем считать, что функция
выигрыша имеет вид \eqref{eq:multidem_payoff_function}.
Обозначим через $S_a(\pi \textbackslash \pi^a)$, где 
$a \in A$ множество Слейтера задачи 
$
	\max \limits_{\pi \in P \: : \: \pi^a \in P^a} \overline F_a(\pi) 
$
где $F_a$ имеет вид \eqref{eq:multidem_payoff_function}.

\begin{Def}
	Решением игры \eqref{eq:ext_normal_form_game} согласно \cite{blackwell} 
	является множество ситуаций 
	\begin{equation}
		P^* = \{ 
			\pi = (\pi^1, \ldots, \pi^m) \in P = \Motimes_{a \in A} P^a \: | \:
			\pi^a \in S_a(\pi \textbackslash \pi^a), \; a \in A
		\}
		\label{eq:game_solution}
	\end{equation}
\end{Def}

В случае конечных многокриетриальных игр Шепли свел \cite{shapley}
описание данного множества к семейству задач поиска значений 
скалярных игр с 
функциями выигрышей -- ЛС частных критериев при произвольном наборе
весовых коэффициентов, своих у каждого игрока.
В случае скалярной игры осреднение однозначно, а  для скаляризованной
вектор-функции могут быть разные варианты.

Целью данной выпускной квалификационной работы является исследование 
применения разных сверток в игре с векторным выигрышем.