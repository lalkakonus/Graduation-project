\subsection{Рассмотрим игру за студента}

Игрок \textbf{С} стремится максимизировать вектор-функцию выигрыша: 
$$
	F(x, y) = (\frac{y\sqrt{x}}{2},\frac{\sqrt{1-x}}{y})
$$
Игрок \textbf{С} использует смешанную стратегию, т.е. его стратегией является
распределение $q=(q_0,q_1,q_2)$ над множеством чистых стратегий 
$X=\{0,\frac{1}{2},1\}$:
\begin{gather*}
	F_\textrm{C}=
	\big \langle
		q_0\frac{y\sqrt{0}}{2} + 
		q_1\frac{y}{2} \sqrt{\frac{1}{2}} + 
		q_2\frac{y\sqrt{1}}{2};
		q_0\frac{\sqrt{1}}{y} +
		q_1\frac{1}{y} \sqrt{\frac{1}{2}} +
		q_2\frac{\sqrt{0}}{y}
	\big \rangle = 
	\\
	=\big \langle
		\frac{y}{2}\cdot\frac{1-q_0+(\sqrt{2}-1)q_2}{\sqrt{2}};
		\frac{1}{y}\cdot\frac{1-q_2+(\sqrt{2}-1)q_0}{\sqrt{2}}
	\big \rangle
\end{gather*}
Затем использует \textbf{ОЛС} \eqref{eq:RL_scalarization}. 
Сначала рассмотрим вырожденные случаи для свёртки - когда 
параметры равны $\mu=0$ и  $\mu=1$. 

%-----------------------------------------------------------------------

\textbf{(1)} 
Если $\mu=0$:
$$
	G(y,q,0)=\frac{1}{\sqrt{2}}\cdot \frac{1-q_2+(\sqrt{2}-1)q_0}{y}
$$	

Далеем осредняем функцию $G(y,q,0)$ по стратегиям 
противника $y \in Y=\{1,2\}$ c вероятностями $(1-p,p)$:
\begin{gather*}
	\overline G(p,q,0)=
	\frac{1-p}{\sqrt{2}} \cdot \frac{1-q_2+(\sqrt{2}-1)q_0}{1}+
	\frac{p}{\sqrt{2}} \cdot \frac{1-q_2+(\sqrt{2}-1)q_0}{2}=\\
	=\frac{(2-p)(1-q_2+(\sqrt{2}-1)q_0)}{2\sqrt{2}}
\end{gather*}
	
Найдём частные производную по переменным $q_0$ и $q_2$. 
Введём следующие обозначения для сокращения записи:

\begin{equation}
	\frac{\partial \overline G(p,q,\mu)}{\partial q}=
	\langle 
		\frac{\partial \overline G(p,q,\mu)}{\partial q_0};
		\frac{\partial \overline G(p,q,\mu)}{\partial q_2} 
	\rangle=
	\langle g_1(p,q,\mu), g_2(p,q,\mu) \rangle
	\label{eq:G_dericative}
\end{equation}

$$
	\frac{\partial \overline G(p,q,0)}{\partial q}
	=\frac{2-p}{2\sqrt{2}} \langle \sqrt{2}-1;-1\rangle
$$
Поскольку $p \leqslant 1$, то $\dfrac{2-p}{2\sqrt{2}} > 0 $.
Мы рассматриваем задачу максимизации \\ на множестве $Q$, cледовательно:
$$
	q^* = \arg \max \limits_{q \in Q} \overline G(p,q,0)=(1,0).
$$
	
%--------------------------------------------------------------------
	
\textbf{(2)}
Если $\mu=1:$
$$
	G(y,q,1)=\frac{1}{\sqrt{2}}\cdot \frac{1-q_2+(\sqrt{2}-1)q_0}{y}
$$	
Далеем осредняем функцию $G(y,q,1)$ по стратегиям 
противника $y \in Y=\{1,2\}$ c вероятностями $(1-p,p)$:
\begin{gather*}
	\overline G(p,q,1)=
	\frac{1-p}{\sqrt{2}} \cdot \frac{1-q_0+(\sqrt{2}-1)q_2}{2}+
	\frac{p}{\sqrt{2}} \cdot \frac{1-q_0+(\sqrt{2}-1)q_2}{1}=\\
	=\frac{(p+1)(1-q_0+(\sqrt{2}-1)q_2)}{2\sqrt{2}}
\end{gather*}
Найдём частные производную по переменным $q_0$ и $q_2$:	
$$
	\frac{\partial \overline G(p,q,1)}{\partial q}
	=\frac{p+1}{2\sqrt{2}} \langle -1; \sqrt{2}-1\rangle
$$
Поскольку $p \geqslant 0$, то $\dfrac{p+1}{2\sqrt{2}} > 0 $.
Мы рассматриваем задачу максимизации на множестве $Q$, 
cледовательно:
$$
	q^* = \arg \max \limits_{q\in Q} \overline G(p,q,1)=(0,1).
$$

%--------------------------------------------------------------------

\textbf{(3)} 
Теперь $\mu \neq 0,1$:
$$
	G(y,q,\mu)=\frac{1}{\sqrt{2}}\min \limits_{0<\mu<1}
	\big \langle
		\frac{y}{2} \cdot \frac{1-q_0+(\sqrt{2}-1)q_2}{\mu};
		\frac{1}{y} \cdot \frac{1-q_2+(\sqrt{2}-1)q_0}{1-\mu}
	\big \rangle	
$$
Далеем осредняем функцию $G(y,q,\mu)$ по стратегиям 
противника $y \in Y=\{1,2\}$ c вероятностями $(1-p,p)$:
\begin{multline}
	\label{eq:G_aver}
	\overline G(p,q,\mu)=\frac{1-p}{\sqrt{2}}\min \limits_{0<\mu<1}
	\big \langle
		\frac{1-q_0+(\sqrt{2}-1)q_2}{2\mu};
		\frac{1-q_2+(\sqrt{2}-1)q_0}{1-\mu}
	\big \rangle + \\
	+\frac{p}{\sqrt{2}}\min \limits_{0<\mu<1}
	\big \langle
		\frac{1-q_0+(\sqrt{2}-1)q_2}{\mu};
		\frac{1-q_2+(\sqrt{2}-1)q_0}{2(1-\mu)}
	\big \rangle 
\end{multline}
	
Введём вспомогательные обозначения:
\begin{gather*}
	\ell_1(q,\mu)=\frac{1-q_0+(\sqrt{2}-1)q_2}{2\mu} \\
	\ell_2(q,\mu)=\frac{1-q_2+(\sqrt{2}-1)q_0}{1-\mu} \\
	\ell_3(q,\mu)=\frac{1-q_0+(\sqrt{2}-1)q_2}{\mu} \\
	\ell_4(q,\mu)=\frac{1-q_2+(\sqrt{2}-1)q_0}{2(1-\mu)}
\end{gather*}
	
	
Для различных значений переменной $\mu$ рассмотрим 
взаимные расположения множеств $\ell_1>\ell_2$ и $\ell_3>\ell_4$
на плоскости $(q_0,q_2)$. Другими словами для фиксированного 
значения $\mu \in [0,1]$ найдём области плоскости, в которых 
достигается минимум одного из выражений в \eqref{eq:G_aver}
	
\begin{figure}[H]
   	\centering
    \begin{subfigure}[b]{0.22 \textwidth}
    	\centering
    	\includegraphics[width=\textwidth]{part_3/graf_6_1}
		\caption{$\mu=0.8$}
        \label{fig:y equals x}
    \end{subfigure}
    \hfill
    \begin{subfigure}[b]{0.22 \textwidth}
       	\centering
       	\includegraphics[width=\textwidth]{part_3/graf_6_2}
       	\caption{$\mu=\frac{2}{3}$}
       	\label{fig:three sin x}
     \end{subfigure}
     \hfill
     \begin{subfigure}[b]{0.22 \textwidth}
       	\centering
       	\includegraphics[width=\textwidth]{part_3/graf_6_3}
       	\caption{$\mu=\frac{1}{3}$}
       	\label{fig:five over x}
     \end{subfigure}
     \hfill
     \begin{subfigure}[b]{0.22 \textwidth}
       	\centering
       	\includegraphics[width=\textwidth]{part_3/graf_6_4}
		\caption{$\mu=0.1$}
        \label{fig:five over x}
     \end{subfigure}
\end{figure}
Поясним график. Синяя область -- это множество $\ell_1 > \ell_2$.
Зелёная область на графике -- это множнство $\ell_3 < \ell_4$.
Область между ними -- это множество $\{\ell_1<\ell_2 \cap \ell_3 > \ell_4\}$.
Исходя из графиков $\{\ell_1>\ell_2 \cap \ell_3 < \ell_4\} = \emptyset$ 
при $(\mu, q) \in (0, 1) \times [0, 1]^2$.
Поскольку граничные случаи для параметра $\mu$ были рассмотрены в первых
двух пунктах, то квадрат $[0,1]^2$ на плоскости $(q_0,q_2)$ делится на 3
связные, не пересекающихся множества.
	
\textbf{Добавить доказательство того, что всего три возможных варианта}\\
\textbf{(1)} 
Рассмотрим полуплоскость, которая определяется системой
$
	\begin{cases}
		\ell_1 > \ell_2 \\
		\ell_3 > \ell_4
	\end{cases}.
$
Вторая строчка в системе является избыточной, т.к. следует из первой.
Система эквивалентна неравенству $\ell_1 > \ell_2$. Выражение \eqref{eq:G_aver} на этом множестве принимает следующий вид:
\begin{gather*}
	\overline G(p,q,\mu)=
	\frac{1-p}{\sqrt{2}} \cdot \frac{1-q_2+(\sqrt{2}-1)q_0}{1-\mu} + 
	\frac{p}{\sqrt{2}} \cdot \frac{1-q_2+(\sqrt{2}-1)q_0}{2(1-\mu)} = \\
	=\frac{2-p}{2\sqrt{2}}\cdot\frac{1-q_2+(\sqrt{2}-1)q_0}{1-\mu}		
\end{gather*}
\textbf{(2)}
Рассмотрим полуплоскость, которая определяется системой
$
	\begin{cases}
		\ell_1 < \ell_2 \\
		\ell_3 < \ell_4
	\end{cases}.
$
Первая строчка в системе является избыточной, т.к. следует из второй.
Система эквивалентна неравенству $\ell_3 < \ell_4$.
Выражение \eqref{eq:G_aver} на этом множестве принимает следующий вид
\begin{gather*}
	\overline G(p,q,\mu)=
	\frac{1-p}{\sqrt{2}} \cdot \frac{1-q_0+(\sqrt{2}-1)q_2}{2\mu} + 
	\frac{p}{\sqrt{2}} \cdot \frac{1-q_0+(\sqrt{2}-1)q_2}{\mu} = \\
	=\frac{1+p}{2\sqrt{2}}\cdot\frac{1-q_0+(\sqrt{2}-1)q_2}{\mu}		
\end{gather*}
\textbf{(3)}
Рассмотрим область, которая определяется системой
$
	\begin{cases}
		\ell_1 < \ell_2 \\
		\ell_3 > \ell_4
	\end{cases}.
$
Выражение \eqref{eq:G_aver} на этом множестве принимает следующий вид:	
\begin{gather*}	
	\overline G(p,q,\mu)=
	\frac{1-p}{\sqrt{2}} \cdot \frac{1-q_0+(\sqrt{2}-1)q_2}{2\mu} +
	\frac{p}{\sqrt{2}} \cdot \frac{1-q_2+(\sqrt{2}-1)q_0}{2(1-\mu)}=
	\\	
	=\frac{(1-p)(1-\mu)(1-q_0+(\sqrt{2}-1)q_2)+p\mu(1-q_2+(\sqrt{2}-1)q_0)}
	{2\sqrt{2}\mu(1-\mu)}
\end{gather*}
	
Итого: 

$$
	\overline G(p,q,\mu) =		
	\begin{cases}
		\dfrac{2-p}{2\sqrt{2}}\cdot\dfrac{1-q_2+(\sqrt{2}-1)q_0}{1-\mu} 
		,\hspace{2mm} \ell_1 \geqslant \ell_2		
		\\
		\dfrac{1+p}{2\sqrt{2}}\cdot\dfrac{1-q_0+(\sqrt{2}-1)q_2}{\mu}
		,\hspace{2mm} \ell_3 \leqslant \ell_4\\
		\dfrac{(1-p)(1-\mu)(1-q_0+(\sqrt{2}-1)q_2)+p\mu(1-q_2+(\sqrt{2}-1)q_0)}
		{2\sqrt{2}\mu(1-\mu)}
		, \textrm{ иначе}
	\end{cases}
$$	
Тогда вектор производных по переменным $(q_0,q_2)$ имеет вид:
$$
	\frac{\partial \overline{G}(p,q,\mu)}{\partial q}=
	\begin{cases}
		\dfrac{2-p}{2\sqrt{2}(1-\mu)} \langle \sqrt{2}-1; -1 \rangle 
 		,\hspace{2mm}
 		\ell_1 \geqslant \ell_2\\
		
		\dfrac{1+p}{2\sqrt{2}\mu} \langle -1; \sqrt{2}-1 \rangle
		,\hspace{2mm}
		\ell_3 \leqslant \ell_4\\
		\textbf{g}(p,\mu)
		,\hspace{2mm}
		\begin{cases}
			\ell_1 \leqslant \ell_2\\
			\ell_3 \geqslant \ell_4\\
		\end{cases}
	\end{cases}
$$
где:
$$
	\textbf{g}(p,\mu) =
	\dfrac{1}{2\sqrt{2}\mu(1-\mu)}
	\big \langle 
		(\sqrt{2} - 1)p\mu -(1-p)(1-\mu);
		(\sqrt{2} - 1)(1-p)(1-\mu) - p\mu			
	\big \rangle
$$
	
%------------------------------------------------------------

\textbf{(1)}
Рассмотрим $\ell_1 \geqslant \ell_2$. Тогда вектор производных 
по переменным $(q_0,q_2)$ имеет вид:
$$
	\frac{\partial \overline{G}(p,q,\mu)}{\partial q}=
	\frac{2-p}{2\sqrt{2}(1-\mu)} \langle \sqrt{2}-1; -1 \rangle
$$	
Введём вспомогалетльную функцию 
$\ell_B(q, \mu):=(\ell_1-\ell_2) \cdot \mu(1-\mu)$,
множество, на котором она принимает неотрицательные значения 
составляют интересующую нас область. Множитель $\mu(1-\mu)$ является строго
положительным на $\mu \in (0,1)$, поэтому не влияет на знак. Функция
является линейной по переменным $q_0$ и $q_2$:
 
$$
	\ell_B(q, \mu) = 
	(1+\mu(2\sqrt{2}-3))q_0+
	(1-\sqrt{2}+\mu(\sqrt{2}-3))q_2
	+3\mu-1
$$ 	
Параметр $\mu$ изменяется в диапазоне $(0,1)$, причём:
\begin{gather*}	
	\ell_B(q,\mu) \xrightarrow[\mu\rightarrow 1]{} 
	1-q_0+(\sqrt{2}-1)q_2\\	
	\ell_B(q,\mu) \xrightarrow[\mu\rightarrow 0]{} 	
	1-q_2+(\sqrt{2}-1)q_0
\end{gather*}
Предельные положения $\ell_B(q, \mu)=0$ изображены на графике 
\eqref{fig:l_B_limits}.

\begin{figure}[H]
	\centering
  	\includegraphics[scale=0.5]{part_3/graf_3_2}
  	\caption{}
  	\label{fig:l_B_limits}
\end{figure}
Найдём значение $\mu$, при котором прямая $\ell_B(q, \mu)$ проходит через точку
$q=(0,0)$:
$$
	\ell_B(0,0,\hat \mu) = 0
	\Rightarrow \hat \mu = \frac{1}{3}
$$	
	
Очевидно, что на полиэдре $P_B(\mu):$
		
$$
	P_B(\mu)=\{q \in Q \; | 
	\;  \ell_B(q, \mu) \geqslant 0 \}, \; \mu \in (0,1),
$$
	
функция $\overline{G}(p,q,\mu)$ достигает максимума в точке $B(\mu):$
$$
	q^* = \arg \max \limits_{q\in P_B(\mu)} \overline G(y,q,\mu) = B(\mu)=
	\begin{cases}
		(0, q_2) : \ell_B(0,q_2,\mu)=0, & \frac{1}{3} \leqslant \mu < 1 \\
		(q_0, 0) : \ell_B(q_0,0,\mu)=0, & 0 < \mu \leqslant \frac{1}{3} \\
	\end{cases}	
$$

\begin{figure}[H]
   	\centering
	\begin{subfigure}[b]{0.4 \textwidth}
        \centering
        \includegraphics[width=\textwidth]{part_3/graf_3_3}
        \caption{$\mu=0.1 < \frac{1}{3}$}
         \label{fig:y equals x}
     \end{subfigure}
     \hspace{10mm}
     \begin{subfigure}[b]{0.4 \textwidth}
       	\centering
       	\includegraphics[width=\textwidth]{part_3/graf_3_4}
       	\caption{$\mu=0.4 > \frac{1}{3}$}
       	\label{fig:three sin x}
     \end{subfigure}
\end{figure}	

На графике зелёным цветом изображена область $P_B(\mu)$, чёрным цветом
обозначены линии уровня и стрелкой соответсвенно градиент функции.
Рассмотрим эти два случая и найдём явное выражение для точки $B(\mu)$:
	
\textbf{(a)}
Если $\frac{1}{3} \leqslant \mu \leqslant 1$ то координата $q_2$ 
точки максимума определяется из условия: 	
\begin{gather*}
	\ell_B(0,q_2,\mu)=0
	\\
	(1-\sqrt{2}+\mu(\sqrt{2}-3))q_2+3\mu-1=0
	\hspace{3mm} \Rightarrow \hspace{3mm}
	q_2=\frac{3\mu-1}{\sqrt{2}-1+(3-\sqrt{2})\mu}	
	\\
	q^*=(0;\dfrac{3\mu-1}{\sqrt{2}-1+(3-\sqrt{2})\mu}), \hspace{3mm}
	\frac{1}{3} \leqslant \mu \leqslant 1
\end{gather*}

\textbf{(b)}
Если $0 \leqslant \mu \leqslant \frac{1}{3}$ то координата $q_0$ точки 
максимума определяется из условия: 	
\begin{gather*}
	\ell_B(q_0,0,\mu)=0
	\\	
	(1+\mu(2\sqrt{2}-3))q_0
	+3\mu-1=0
	\hspace{3mm} \Rightarrow \hspace{3mm}
	q_0=\frac{1-3\mu}{1+(2\sqrt{2}-3\mu)}	
	\\ 	
	q^*=(\frac{1-3\mu}{1+(2\sqrt{2}-3\mu)}; 0), \hspace{3mm}
	0 \leqslant \mu \leqslant \frac{1}{3}
\end{gather*}
Следовательно:
\begin{equation} 
	\label{eq:B_point}
	B(\mu) = \arg \max \limits_{q\in P_B(\mu)} \overline G(y,q,\mu) = 
	\begin{cases}
		(0;\dfrac{3\mu-1}{\sqrt{2}-1+(3-\sqrt{2})\mu})
		, & \frac{1}{3} \leqslant \mu \leqslant 1
		\\
		(\dfrac{1-3\mu}{1+(2\sqrt{2}-3\mu)};0)
		, & 0 \leqslant \mu \leqslant \frac{1}{3}
	\end{cases}
\end{equation}

%------------------------------------------------------------

\textbf{(2)} Рассмотрим $\ell_3 \leqslant \ell_4$.
Производная в области имеет вид:
	
$$
	\frac{\partial \overline{G}(p,q,\mu)}{\partial q}=
	\frac{1+p}{2\sqrt{2}\mu} \langle -1;\sqrt{2}-1 \rangle 
$$
 	
Введём вспомогалетльную функцию	
$\ell_A(q, \mu):=(\ell_3-\ell_4) \cdot \mu(1-\mu)$,
множество, на котором она принимает неотрицательные значения 
составляют интересующую нас область. Множитель $\mu(1-\mu)$ является строго
положительным на $\mu \in (0,1)$, поэтому не влияет на знак. Функция
является линейной по переменным $q_0$ и $q_2$:
 	
$$
	\ell_A(q, \mu)=
	-(2+(\sqrt{2}-3)\mu)q_0
	-(2-2\sqrt{2}+(2\sqrt{2}-3)\mu)q_2
	-3\mu+2
$$ 	
 	
Параметр $\mu$ изменяется в диапазоне $(0,1)$, причём	
	
\begin{gather*}	
	\ell_A(q,\mu) \xrightarrow[\mu\rightarrow 1]{} 
	1-q_0+(\sqrt{2}-1)q_2\\	
	\ell_A(q,\mu) \xrightarrow[\mu\rightarrow 0]{} 	
	1-q_2+(\sqrt{2}-1)q_0\\
\end{gather*}
	
	Предельные положения $\ell_B(q, \mu)=0$ изображены на графике \eqref{fig:l_B_limits}.
	Найдём значение $\mu$ при котором прямая 
	$\ell(q, \mu)$ проходит через точку
	$q=(0,0)$:
	
	$$\ell_A(0,0,\hat \mu) = 0 \Rightarrow \hat \mu = \frac{2}{3}$$
	
	Очевидно, что на полиэдре $P_2(\mu):$
	
	$$P_A(\mu)=\{q \in Q \; | 
	\;  \ell_3(q, \mu) \leqslant \ell_4(q, \mu) \}, \; \mu \in (0,1)$$
	
	функция $\overline{G}(p,q,\mu)$ достигает максимума в точке $A(\mu):$
	
	\begin{gather*}
		A(\mu)= \arg \max \limits_{q\in P_A(\mu)} \overline G(p,q,\mu) =
		\begin{cases}
			(0, q_2) : \ell_A(0,q_2,\mu)=0, & \frac{2}{3} \leqslant \mu \leqslant 1 \\
			(q_0, 0) : \ell_A(q_0,0,\mu)=0, & 0 \leqslant \mu \leqslant \frac{2}{3} \\
		\end{cases}		
	\end{gather*}
	
	\begin{figure}[H]
    	\centering
     	\begin{subfigure}[b]{0.45 \textwidth}
        	\centering
        	\includegraphics[width=\textwidth]{part_3/graf_3_5}
        	\caption{$\mu=0.1 < \frac{2}{3}$}
         	\label{fig:y equals x}
     	\end{subfigure}
     	\hspace{10mm}
     	\begin{subfigure}[b]{0.45 \textwidth}
        	\centering
        	\includegraphics[width=\textwidth]{part_3/graf_3_6}
        	\caption{$\mu=0.8 > \frac{2}{3}$}
        	\label{fig:three sin x}
     	\end{subfigure}
	\end{figure}	
	
	На графике зелёным цветом изображена область $P_A(\mu)$, чёрным цветом
	обозначены линии уровня и стрелкой соответсвенно градиент функции.
	Рассмотрим эти два случая и найдём явное выражение для точки $A(\mu)$:

	\textbf{(a)} Если $\frac{2}{3} \leqslant \mu \leqslant 1$ то координата $q_2$ точки 
	максимума определяется из условия: 	
	
	\begin{gather*}	
		\ell_A(0,q_2,\mu)=0 
		\\
		-(2-2\sqrt{2}+(2\sqrt{2}-3)\mu)q_2
		-3\mu+2=0
		\hspace{3mm} \Rightarrow \hspace{3mm}
		q_2=\frac{3\mu-2}{2(\sqrt{2}-1)+(3-2\sqrt{2})\mu}	
		\\
		q^*= (0;\dfrac{3\mu-2}{2(\sqrt{2}-1)+(3-2\sqrt{2})\mu}), 
		\hspace{5mm} \frac{2}{3} \leqslant \mu \leqslant 1
	\end{gather*}


	\textbf{(b)} Если $0 \leqslant \mu \leqslant \frac{2}{3}$ то координата $q_0$ точки 
	максимума определяется из условия: 	
	
	\begin{gather*}
		\ell_A(q_0,0,\mu)=0	
		\\
		-(2+(\sqrt{2}-3)\mu)q_0
		-3\mu+2=0	
		\hspace{3mm} \Rightarrow \hspace{3mm}
		q_0=\frac{2-3\mu}{2+(\sqrt{2}-3)\mu}	
		\\
		q^*= (\dfrac{2-3\mu}{2+(\sqrt{2}-3)\mu};0), 
		\hspace{5mm} 0 \leqslant \mu \leqslant \frac{2}{3}	
	\end{gather*}
	
	Следовательно:
	\begin{equation} \label{eq:A}
		A(\mu) = \arg \max \limits_{q\in P_A(\mu)} \overline G(p,q,\mu) = 
		\begin{cases}
			(0;\dfrac{3\mu-2}{2(\sqrt{2}-1)+(3-2\sqrt{2})\mu})
			, & \frac{2}{3} \leqslant \mu \leqslant 1 \\
			(\dfrac{2-3\mu}{2+(\sqrt{2}-3)\mu};0), & 0 \leqslant \mu \leqslant \frac{2}{3}	
		\end{cases}
	\end{equation}

%------------------------------------------------------------

	\textbf{(3)} Рассмотрим область в которой
	$\begin{cases}
		\ell_1 \leqslant \ell_2 \\	
		\ell_3 \geqslant \ell_4 \\
	\end{cases}	
	$
	
	Частные производные имеют следующий види в данной области.
	
	$
	\dfrac{\partial \overline{G}(p,q,\mu)}{\partial q}=
	\dfrac{1}{2\sqrt{2}\mu(1-\mu)}
	\big \langle 
		(\sqrt{2} - 1)p\mu -(1-p)(1-\mu);
		(\sqrt{2} - 1)(1-p)(1-\mu) - p\mu			
	\big \rangle
	$

	\begin{gather*}
	g_1=(\sqrt{2} - 1)p\mu -(1-p)(1-\mu) \\
	g_2=(\sqrt{2} - 1)(1-p)(1-\mu) - p\mu
	\end{gather*}

 	Заметим, что функция $\overline{G}(p,q,\mu)$
 	является линейной по переменным $q_0$ и $q_2$ т.е.:

	\begin{gather*}
	\overline{G}(p,q,\mu)=g_0(p,\mu) \: q_0+g_2(p,\mu) \: q_2+c(p,\mu)
	\\
	\textrm{и}
	\\
	\overline{G}(p,q,\mu) = 0 \sim q_2 = k(p, \mu) \cdot q_0 + c(p, \mu)	
	\end{gather*}
	
	Рассматриваем функцию на полиэдре $P_{AB}(\mu):$
	
	$$P_{AB}(\mu)=
	\{
		q \in Q \; | \;  
		\ell_A(q, \mu) \geqslant 0 \cap
	 	\ell_B(q, \mu) \leqslant 0
	\},\; \mu \in (0,1) $$

	Ограничение $\ell_A(q, \mu) = 0$ и $\ell_B(q, \mu) = 0$ представимы в 
	эквивалентном виде:
	
	$$\ell_A(q,\mu)=0 \sim \; q_2=k_A(\mu)q_0+c_A(\mu)$$
	$$\ell_B(q,\mu)=0 \sim \; q_2=k_B(\mu)q_0+c_B(\mu)$$

	
	\begin{figure}[H]
		\centering
  		\includegraphics[scale=0.3]{part_3/graf_3_10}
  		\caption{Коэффициенты $k_A(\mu)$, $k_B(\mu)$ и $k(p,\mu)$ при фиксированном $p$}
		\label{fig:k_A,k_B,k}	
	\end{figure}	
	
	Рассмотрим график \ref{fig:k_A,k_B,k}, на котором изображены значения коэффициетов
	при фиксированном значении $p \in [0,1]$. $k_B(\mu)$ изображены красной кривой,
	$k_A(\mu)$ синей и фиолетовой - кривая $k(p, \mu)$. Пунктирная вертикальная линия
	обозначает точку $x$ в которой выражение $g_1=0$. Имеет место неравенство 
	$k_A(\mu) > k_B(\mu)$ при $\mu \in (0,1)$.
	
	$$
	\begin{cases}
	k(\mu) < k_B(\mu) \textrm{ при } g_1(\mu,p) < 0  \\
	k(\mu) > k_A(\mu) \textrm{ при } g_1(\mu,p) > 0 \\
	\end{cases}		
	$$	
	
	
	Следовательно точки максимума функции $\overline{G}(p,q,\mu)$ при фикисированных
	значениях $\mu$ и $p$ могут быть точки: 
	$\{A\}, \{B\}, (0,0)$ и отрезки $[B, A], [0, A], [0,B]$. Где точки
	$A(\mu)$ и $B(\mu)$ определны в \ref{eq:A} и \ref{eq:B}. 
	Рассмотрим три подслучая:
	
	\newpage

	\textbf{(a)} $0 \leqslant \mu \leqslant \frac{1}{3}$. 	
		
	\begin{figure}[H]
    	\centering
     	\begin{subfigure}[b]{0.3 \textwidth}
        	\centering
        	\includegraphics[width=\textwidth]{part_3/graf_3_9_0}
        	\caption{$g_1 > 0 \Rightarrow q^*=A$}
         	\label{fig:y equals x}
     	\end{subfigure}
     	\begin{subfigure}[b]{0.3 \textwidth}
        	\centering
        	\includegraphics[width=\textwidth]{part_3/graf_3_9_2}
        	\caption{$g_1 = 0 \Rightarrow q^*=[B,A]$}
        	\label{fig:three sin x}
     	\end{subfigure}
     	\begin{subfigure}[b]{0.3 \textwidth}
        	\centering
        	\includegraphics[width=\textwidth]{part_3/graf_3_9_1}
        	\caption{$g_1 < 0 \Rightarrow q^*=B$}
        	\label{fig:three sin x}
     	\end{subfigure}
     	\caption{}
     	\label{fig:3_mu_0}
	\end{figure}

	%Если наклон касательной меньше $k_A$ но больше нуля, то оптимальной точкой
	%является точка $A(\mu)$, если касатальеная паралельна оси $Oq_0$,
	%то множеству оптимальных точек соответсвует отрезок $[B,A]$, если же наклон
	%отрицательный то оптимальной точкой является точка $B(\mu)$.
	%Итого оптимальные точки соответсвуют следующей системе:	
	
	$$
		\begin{cases}
			g_1 > 0 \Rightarrow q^*=A \\
			g_1 = 0 \Rightarrow q^*=[B,A] \\
			g_1 < 0 \Rightarrow q^*=B
		\end{cases}
	$$	
	
	\textbf{(b)} $\frac{2}{3} \leqslant \mu \leqslant 1$
			
	\begin{figure}[H]
    	\centering
     	\begin{subfigure}[b]{0.3 \textwidth}
        	\centering
        	\includegraphics[width=\textwidth]{part_3/graf_3_7_0}
        	\caption{$g_2 > 0 \Rightarrow q^*=B$}
         	\label{fig:y equals x}
     	\end{subfigure}
     	\begin{subfigure}[b]{0.3 \textwidth}
        	\centering
        	\includegraphics[width=\textwidth]{part_3/graf_3_7_1}
        	\caption{$g_2 = 0 \Rightarrow q^*=[B,A]$}
        	\label{fig:three sin x}
     	\end{subfigure}
     	\begin{subfigure}[b]{0.3 \textwidth}
        	\centering
        	\includegraphics[width=\textwidth]{part_3/graf_3_7_2}
        	\caption{$g_2 < 0 \Rightarrow q^*=A$}
        	\label{fig:three sin x}
     	\end{subfigure}
     	\caption{}
	\end{figure}
	
	\begin{center}
		$\left[
		\begin{gathered}
			g_2 > 0 \Rightarrow q^*=B \\
			g_2 = 0 \Rightarrow q^*=[B,A] \\
			g_2 < 0 \Rightarrow q^*=A
		\end{gathered}
		\right.$	
	\end{center}	
	
	\newpage

	\textbf{(c)} $\frac{1}{3} \leqslant \mu \leqslant \frac{2}{3}$
	
	\begin{figure}[H]
    	\centering
     	\begin{subfigure}[b]{0.3 \textwidth}
        	\centering
        	\includegraphics[width=\textwidth]{part_3/graf_3_8_1}
        	\caption{$g_2 > 0 \Rightarrow q^*=B$}
     	\end{subfigure}
     	\begin{subfigure}[b]{0.3 \textwidth}
        	\centering
        	\includegraphics[width=\textwidth]{part_3/graf_3_8_2}
        	\caption{$g_1 > 0 \Rightarrow q^*=A$}
     	\end{subfigure}
     	\begin{subfigure}[b]{0.3 \textwidth}
        	\centering
        	\includegraphics[width=\textwidth]{part_3/graf_3_8_0}
        	\caption{$
        	\begin{cases}			
				g_2 < 0 \\
				g_1 < 0
			\end{cases}	\Rightarrow q^*=(0,0)
			$}
     	\end{subfigure}
    	\centering
     	\begin{subfigure}[b]{0.3 \textwidth}
        	\centering
        	\includegraphics[width=\textwidth]{part_3/graf_3_8_3}
        	\caption{$g_2 = 0 \Rightarrow q^*=[0,B]$}
     	\end{subfigure}
     	\begin{subfigure}[b]{0.3 \textwidth}
        	\centering
        	\includegraphics[width=\textwidth]{part_3/graf_3_8_4}
        	\caption{$g_1 = 0 \Rightarrow q^*=[0,A]$}
     	\end{subfigure}
	\end{figure}


	
	\begin{center}
		$\left[
		\begin{gathered}
			g_1 > 0 \Rightarrow q^*=A \\
			g_2 > 0 \Rightarrow q^*=B \\			
			g_1 = 0 \Rightarrow q^*=[0,A] \\
			g_2 = 0 \Rightarrow q^*=[0,B] \\			
			\begin{cases}			
				g_2 < 0 \\
				g_1 < 0
			\end{cases}	\Rightarrow q^*=(0,0)
		\end{gathered}
		\right.$	
	\end{center}

%-------------------------------------------------------------

	Итого получим следующие 

	$C(p,\mu) = \arg \max \limits_{P_{AB}} \overline{G}(p,q,\mu)
	\begin{cases}
	A(\mu), & g_2(p,\mu)<0 \cup \mu \leqslant \frac{1}{3} \cap g_2 < 0 \\
	B(\mu), & g_1(p,\mu)>0 \cup \mu \geqslant \frac{2}{3} \cap g_1 > 0 \\
	(0,0),  & g_1(p,\mu)<0 \cap g_2(p,\mu)<0 \\
	[A(\mu), B(\mu)], 
	& g_1(\mu)=0 \cap \mu < \frac{1}{3} \cup g_1(\mu)=0 \cap \mu > \frac{2}{3} \\
	[0, A(\mu)], & g_2(\mu)=0 \cap \mu \in [\frac{1}{3},\frac{2}{3}] \\
	[0, B(\mu)], & g_1(\mu)=0 \cap \mu \in [\frac{1}{3},\frac{2}{3}] \\
	\end{cases}	
	$	
	
	Но кроме того точки $A(\mu)$ и $B(\mu)$ являются оптимальными 
	$\forall$ $p$ и $\mu$ поскольку	являются таковыми в \textbf{(1)} и \textbf{(2)}.
	Итого в области $P_B$ оптимальной является точка $B(\mu)$,
	в области $P_A$ оптимальной является точка $A(\mu)$,
	и в области $P_{AB}$ оптимальной является точка $C(p, \mu)$.
	Поскольку если $f(x)$ и $g(x)$ непрерывны на множестве $X$, то и 
	$min(f(x), g(x))$ непрерывна на $X$, то максимум достигается в точке
	$C$.
	
	$q^*(p,\mu)= \arg \max \limits_Q \overline{G}(p,q,\mu)
	\begin{cases}
	A(\mu), & g_2(p,\mu)<0 \cup \mu \leqslant \frac{1}{3} \cap g_2 < 0 \\
	B(\mu), & g_1(p,\mu)>0 \cup \mu \geqslant \frac{2}{3} \cap g_1 > 0 \\
	(0,0),  & g_1(p,\mu)<0 \cap g_2(p,\mu)<0 \\
	[A(\mu), B(\mu)], 
	& g_1(\mu)=0 \cap \mu < \frac{1}{3} \cup g_1(\mu)=0 \cap \mu > \frac{2}{3} \\
	[0, A(\mu)], & g_2(\mu)=0 \cap \mu \in [\frac{1}{3},\frac{2}{3}] \\
	[0, B(\mu)], & g_1(\mu)=0 \cap \mu \in [\frac{1}{3},\frac{2}{3}] \\
	(0,1), & \mu = 0 \\
	(1,0), & \mu = 1
	\end{cases}	
	$		
	
	
	%$$g_1=p\mu-(1-p)(1-\mu)  \hspace{10mm}
	%g_1=0 \sim p=\frac{1-\mu}{(\sqrt{2}-2)\mu + 1}$$

	%$$g_2=(1-p)(1-\mu)(\sqrt{2}-1)-p\mu  \hspace{10mm}
	%g_2=0 \sim p=\frac{(1-\mu)(\sqrt{2}-1)}{(\sqrt{2}-2)\mu + (\sqrt{2}-1)}$$
	
	Поскольку в пункте 3.1 мы установили, что 
	
	$$p^*(q,\lambda(q))=\arg \max \limits_{p \in [0,1]}
		\overline{L}(p, q, \lambda(q))=[0,1]
	$$

	Нас интересуют оптимальные пары $(p^0,q^0)$ такие, что 
	$\exists (\lambda, \mu) \in [0,1]^2 $:
	
	$$\begin{cases}
	p^0=p^*(q^0,\lambda) \\
	q^0=q^*(p^0,\mu)
	\end{cases}$$
	
	Введём следующие множества:
	
	$$P_0=[0,1], \; Q_0=[0,1] \times \{0\} \cup \{1\} \times [0,1]$$
	
	Следующие точки являются оптимальными:
	$$(p, q) \in P_0 \times Q_0$$













