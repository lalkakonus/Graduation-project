\section{Введение}

\begin{flushleft}

	\textbf{Определение [1]}\\
	Мы рассмотрим \textbf{\textit{линейную свёртку} (ЛС)}:
	
	\begin{equation}
		L(\{f_i\}, \lambda)(x)=\sum_{i=1}^{m} \lambda_i f_i 
		\textrm{, где }
		\lambda \in \Lambda = \{\lambda_i \geqslant 0 \: | \sum_{i=1}^n \lambda_i =1 \},
	\end{equation}

	и \textbf{\textit{свёртку Гермейера}}, или 
	\textbf{\textit{обратную логическую свёртку} (ОЛС)}:
	
	\begin{equation}
		G(\{f_i\}, \lambda)(x)=\min\limits_{\lambda_i > 0} \frac{f_i}{\lambda_i} 
		\textrm{, 	где }
		\lambda \in \Lambda =\{\lambda_i \geq 0 | \sum_{i=1}^n \lambda_i =1 \}.
	\end{equation}

\end{flushleft}