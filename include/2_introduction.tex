\section{Введение}

\qquad В работе рассматриваются различные способы решения модельной задачи которая
представляет собой игру двух лиц с противоположными интересами и двумерной
функцией выигрыша. Для решения задачи применяется модифицированный
метод свёрток предложенный Л.С. Шепли \cite{shapley}, который как правило используется в
подобных задачах.

\subsection{Игровая модель}

\qquad Рассматриваются два игрока - Студент, далее обозначается \textbf{С},
и Преподаватель, далее обозначается \textbf{П}, которые имеют противоположные интересы.
Критерий интересов составляют две велечины, первая из которых является эффективностью
работы \textbf{С} в научной сфере, а второй его эффективностью на подработке.

\textbf{С} выбирает долю $x$ рабочего времени, которую он тратит на подготовку
диплома, оставшееся рабочее время $1-x$ он тратит на подработку. Считается, что производительность \textbf{С} при любых занятий падает с увеличением 
отводимого на них времени, эффективность труда \textbf{С} зададим функцией $\sqrt{x}$
и $\sqrt{1-x}$ соответственно. \textbf{С} может распределять свое время между двумя 
видами деятельности, т.е. имеет множетсво стратегий $x\in X = \{0, 1\}$, 
причём он может использовать смешанные стратегии.

\textbf{П} выбирает - отностится к \textbf{С} требовательно, способствую 
написанию диплома и мешая подработке или же не обращать на него внимания не 
мешая подработке и не помогая с дипломом. \textbf{П} имеет множество стратегий 
$y \in Y=\{1, 2\}$, причём тоже может использовать смешаные стратегии.

\vspace{5mm}
Получаем следующий функциональный критерий:

\begin{equation}
	F(x, y)=
	\big(f_1(x,y), f_2(x,y)\big) =
	\Big(
		\frac{y\sqrt{x}}2,
		\frac{\sqrt{1-x}}y
	\Big)
	\label{eq:player_criterion}
\end{equation}

\qquad \textbf{П} стремится минимизировать (выборая $y \in Y = \{1,2\}$)
критерий $F(x, y)$, а игрок \textbf{С} - максимизировать
 (выбирая $x \in X=[0,1]$).

\vspace{5mm} 
Задачу можно представить в виде многокритериальную игру двух лиц 
с противоположными интересамив 

\begin{equation}
	\bigg \langle F(x,y), X, Y \bigg \rangle, \; y \in Y=\{1, 2\}, \; x \in {X}=[0, 1]
	\label{eq:mc_game}
\end{equation}

\newtheorem{Def}{Определение}
\begin{Def}
	Допустимое решение $\hat{x}\in{X}$ называется 
	строго эффективным (эффективным по Слейтеру) для задачи
	\begin{equation}
		\max\limits_{x \in X} F(x)=({f}_1(x),\ldots, {f}_n(x))
		\label{eq:mc_problem}
	\end{equation}
	если \textbf{не} существует $x\in{X}$ такого, что 
	$f_k(x)>f_k(\hat{x})$ для всех $k=\{1,...,n\}$. Множетсво всех эффективных по
	Слейтеру решений называается \textit{множеством Слейтера} задачи \eqref{eq:mc_problem}.
\end{Def}

Для задачи \eqref{eq:mc_game} введеём следующие частные случаи:

\begin{gather*}
	S_x(y^*) \textrm{ - множество Слейтера задачи } \max\limits_{x \in X} F(x, y^*) \\
	S_y(x^*) \textrm{ - множество Слейтера задачи } \min\limits_{y \in Y} F(x^*, y)
\end{gather*}

\begin{Def}
	Решением\footnote {
	Согласно \textit{Blackwell D.} An analog of the minimax theorem for
    vector payoffs // Pac. J. of Math. 1956. No 6.
	} 
	\eqref{eq:mc_game} является множество точек $(x^*, y^*)$ таких, что
    $x^* \in Х S_x(y^*) $ и $ y^*  \in S_y (x^*)$, 
\end{Def}

\vspace{5mm}

Для параметризации множеств Слейтера будем использовать метод свёрток.
Он заключается в том, что задача $\max\limits_{x \in X} F(x)$ заменяется параметрическим 
семейством скалярных задач $\max\limits_{x \in X} C(\{f_i\}, \lambda, x)$,
где $C$ – функция свертки частных критериев $\{f_i\}_{i=1}^m$ задачи 
\eqref{eq:mc_problem} в единый скалярный критерий, $\lambda$ – параметр свертки. 

\begin{Def}
	Линейной свёрткой с параметром $\lambda$ для функции критериев задачи
	\eqref{eq:mc_problem} называется функция:
	\begin{equation}
		L(\{f_i\}, \lambda, x)=\sum_{i=1}^{m} \lambda_i f_i \textrm{, где }
		\lambda \in \Lambda =\{\lambda_i \geq 0 | \sum_{i=1}^n \lambda_i =1 \},
		\label{eq:linear_scalarization}
	\end{equation}

	\begin{flushleft}
	свёрткой Гермейера или обратной логической свёрткой с параметром $\mu$ для 
	функции критериев задачи \eqref{eq:mc_problem} называется функция:
	\end{flushleft}	
	
	\begin{equation}
		G(\{f_i\}, \mu, x)=
		\min \limits_{i: \mu_i > 0} \frac{f_i}{\mu_i} \textrm{, где }
		\mu \in M =\{\mu_i \geq 0 | \sum_{i=1}^n \mu_i =1 \}.
		\label{eq:germeyer_scalarization}	
	\end{equation}
\end{Def}

В случае конечных $X$ и $Y$ Шепли свел \cite{shapley} описание данного множества к
семейству задач поиска значений скалярных игр с функциями выигрышей – 
ЛС частных критериев при произвольном наборе весовых коэффициентов, 
своих у каждого игрока. Применение линейной свертки в многокритериальных 
задачах обосновывается леммой Карлина

\newtheorem{Th}{Теорема}
\begin{Th}[Карлин \cite{carlin}]
	Пусть $x_0$ – эффективная точка,
    Тогда существуют неотрицательные числа $\lambda_1,…,\lambda_m$ такие, 
    что $\sum_{i=1}^m \lambda_i=1$ и $x_0$ является точкой максимума функции  
    $L(x) =\sum_{j=1}^m \lambda_j f^j(x)$ 
\end{Th}

Гермейером была предложена свертка, которая также аппроксимирует 
множество Слейтера. В работе используется ее модификация -
обратная логическая свертка, она отличается тем, что веса стоят в знаменателе.

\begin{Th}[Гермейер \cite{germeyer}]
    Пусть $x_0$ – эффективная точка, причем $f^i(x_0)>0$ для всех 
    $i=1,…,m.$
    Тогда существуют положительные числа $\lambda_1,…,\lambda_m$ такие, 
    что $\sum_{i=1}^m \lambda_i=1$ и $x_0$ является точкой максимума функции  
    $G(x) =\min \limits_{1 \leqslant j \leqslant m} \lambda_j f^j(x)$.
\end{Th}

Рассмотрим случай, когда \textbf{С} использует обратную логическую свертку,
с парамером $\mu$, a \textbf{П} использует линейную
свертку с параметром $\lambda$. Тогда множество оптимальных решений:

\begin{gather*}
	\begin{cases}
		x^*=\arg \max \limits_{x}   G(\{f_1, f_2\}, \mu, x, y^*) \\
		y^*=\arg \min \limits_{y}   L(\{f_1, f_2\}, \lambda, x^*, y)
	\end{cases}
\end{gather*}

Поскольку игроки используют смешанные стратегии т.е. распределения вероятностей
$\rho_x(x)$ и $\rho_y(y)$ над чистыми стратегиями $x \in X$ и $y \in Y$. 
Далее каждый игрок осредняет свою функцию выигрыша по стратегиям противника

$$ \overline G(\{f_1, f_2\}, \mu, q, p) = 
\iint \limits_{PQ} G(\{f_1, f_2\}, \mu, q, p) \rho_x(x) \rho_y(y)dxdy$$
$$ \overline L(\{f_1, f_2\}, \lambda, q, p) = 
\iint \limits_{PQ} L(\{f_1, f_2\}, \lambda, q, p) \rho_x(x) \rho_y(y)dxdy$$

\begin{Def}	\label{def:optimal_strategy}
	Пара стратегий $(p^0, q^0)$ называется оптимальными, если для некоторых 
	$\lambda$, $\mu$ верно:
	\begin{equation}
		\begin{cases} 
		p^0(q^0, \lambda) = \argmin\limits_{p \in P} \overline L(p, q^0, \lambda) \\ 
		q^0(p^0, \mu) = \argmin\limits_{q \in Q} \overline G(p^0, q, \mu) \\
	\end{cases}
	\end{equation}
\end{Def}

Мы будем рассматривать конечную игру \textbf{C --- П}, полученную из исходной 
заменой множества $X=[0,1]$ конечным множеством точек

$$X^T = \{
	0, \; \frac{1}{T}, \; \frac{2}{T}, \; \ldots, \; \frac{T-1}{T}, \; 1
\}, \;\; T \in \mathbb{N}$$

В работе исследуются случаи $T=1$: $X^1=\{0, 1\}$, 
и $T=2$: $X^2=\{0, \frac{1}{2} ,1\}$