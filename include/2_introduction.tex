\section{Введение}
\begin{flushleft}

Рассмотрим задачу многокритериальной или векторной, оптимизации:
\begin{equation}
\textrm{Найти }\max\limits_x F(x), \textrm{ где } F(x)=({f}_1(x),\ldots, {f}_n(x)),\quad x\in X\textrm{ конечно}
\end{equation}


\textbf{Определение [1]} 

Допустимое решение $\hat{x}\in{X}$ называется \textit{строго эффективным (эффективным по Слейтеру)}, если не существует
$x\in{X}$ такого, что $f(x)>f(\hat{x})$ т.е. $f_k(x)>f_k(\hat{x})$ для всех $k=1,...,n$. Множетсво всех эффективных по
Слейтеру решений называается \textit{множеством Слейтера} задачи (1).
\vspace{5mm}

Для параметризации задачи и поиска множества Слейтера используется метод свёрток, согласно которому МК-задача
$\max\limits_x F(x)$ заменяется параметрическим семейством скалярных задач $\max\limits_x C(\{f_i\}, \lambda)(x)$
где $C$ – функция свертки частных критериев $\{f_i\}_{i=1}^m$ МК-задачи в единый скалярный критерий (монотонна по f), 
$\lambda$ – параметр свертки. 
\vspace{5mm}

Мы рассмотрим \textit{линейную свёртку} (ЛС):
\begin{equation}
L(\{f_i\}, \lambda)(x)=\sum_{i=1}^{m} \lambda_i f_i \textrm{, где }
\lambda \in \Lambda =\{\lambda_i \geq 0 | \sum_{i=1}^n \lambda_i =1 \},
\end{equation}

и \textit{свёртку Гермейера}, или обратную логическую свёртку (ОЛС):
\begin{equation}
G(\{f_i\}, \lambda)(x)=\min\limits_{\lambda_i > 0} \frac{f_i}{\lambda_i} \textrm{, где }
\lambda \in \Lambda =\{\lambda_i \geq 0 | \sum_{i=1}^n \lambda_i =1 \}.
\end{equation}

\textbf{Игровая модель}

Студент (С) выбирает долю $x$ рабочего времени, которую он тратит на подготовку диплома. Допустим, что эффективность подготовки пропорциональна $\sqrt{x}$. Оставшееся рабочее время $1-x$ он тратит на подработку, и его второй 
критерий пропорционален $\sqrt{1-x}$. Считается, что производительность любых занятий С падает с увеличением отводимого на них времени, обусловливая вогнутость по $x$ частных критериев. Допустим, что Студент не может распределять свое время между двумя 
видами деятельности, т.е. имеет множетсво стратегий $x\in X = \{0, 1\}$, причём может использовать смешанные стратегии.
Преподаватель (П) имеет множество стратегий $y \in Y=\{1, 2\}$, причём тоже может использовать смешаные стратегии.
При этом П стремится минимизировать, а С максимизировать следующий векторный критерий: 
\begin{equation}
F(x, y)=(f_1(x, y), f_2(x, y)) =(\frac{y \sqrt{x}}{2}, \frac{\sqrt{1-x}}{y})
\end{equation}
Для введём обозначения для вероятностей:
\[
\begin{cases}
P(X=0)=1-q \\
P(X=1)=q \\
\end{cases}
\textrm{и }
\begin{cases}
P(Y=1)=1-p \\
P(Y=2)=p \\
\end{cases}
\]
И укажем области определения переменных, используемых ниже:
\[
p \in [0, 1],\quad q \in [0, 1],\quad
\mu \in [0, 1],\quad \lambda \in [0, 1]
\]

И предположим, что Студент осредняет свой критерий по переменной $x$ и получает:
$$F(q, y)=(f_1(q, y), f_2(q, y)) =\big(\frac{q y}{2}, \frac{1-q}{y}\big),$$
а Преподователь осредняет свой критерий по переменной $y$ и получает
$$F(x, p)=(f_1(x, p), f_2(x, p)) =\big(\frac{(1+p)\sqrt{x}}{2}, \frac{(2-p)\sqrt{1-x}}{2}\big)$$

\vspace{5mm}

Рассмотрим вариант, когда Студент использует обратную логическую свёртку и осредняет её по переменной $y$:
\begin{equation} 
M(p,q,\mu)=p\min{\{\frac{q}{\mu};\frac{1-q}{2(1-\mu)}\}}+(1-p)\min\{\frac{q}{2\mu};\frac{1-q}{1-\mu}\},
\end{equation}

a Преподаватель использует линейную свёртку и осредняет её по переменной $x$:
\begin{equation} 
L(p,q,\lambda)=\frac{1}{2}\big \{q(3\lambda+p-2)+(2-p)(1-\lambda)\big \}.
\end{equation} 

Игроки пытаются максимизировать или минимизировать сответствующие функции с помощью выбора своей стратегии
\[
\begin{cases} M(p,q,\mu)\rightarrow\max\limits_{q} \\
L(p,q,\lambda)\rightarrow\min\limits_{p}\end{cases}
\]
\textbf{Определение}

Пара стратегий $(p^{*}, q^{*})$ называется \textit{оптимальной}, если для некоторых $\lambda$, $\mu$ верно:
\begin{equation}
\begin{cases} 
p^{*}=p^{*}(q^{*}, \lambda) = \argmin\limits_{p} L(p, q^{*}, \lambda) \\ 
q^{*}=q^{*}(p^{*}, \mu) = \argmin\limits_{q} M(p^{*}, q, \mu) \\
\end{cases}
\end{equation}
\vspace{5mm}

Для исследования оптимальных стратегий необходимо установить значения $p$ и $q$, при которых
эти функции достигают минимума и максимума соответственно. Этот вопрос уже был исследован. Из статьи [3] следует, что:
\begin{equation}
p^{*}(q, \lambda)=\argmin\limits_{p}L(p,q,\lambda)=\begin{cases}0,& q>1-\lambda \\
1,& q<1-\lambda \\
\forall,& q=1-\lambda\end{cases}
\end{equation}

Из курсовой [2] следует, что
\begin{equation}
q^{*}(p, \mu)=\argmax\limits_{q}M(p,q,\mu)=
\begin{cases}
\frac{\mu}{2-\mu},& p+\mu-1\geq 0 \\
\frac{2\mu}{1+\mu},& p+\mu-1\leq 0 \\
\end{cases}
\end{equation}

\end{flushleft}