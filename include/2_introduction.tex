\section{Введение}

\qquad Теорией игр называется математическая теория принятия решений
в конфликтных ситуациях. В простейших моделях рассматривается лицо
принимающее решение (ЛПР), выбирает своё действие из некторогр множества стратегий. Считается, что задана целевая функция,
которая отражает интресы ЛПР и зависит от выбранных им стратегий.
Задача принятия решений состоит в том, чтобы найти стратегию, 
доставляющую максимум целевой функции. Отличие конфликтной ситуации
заключается в том, что решения принимаются не одним лицом, 
а всеми участниками модельной игры и функция выигрыша
кадого индивида зависит не только от его решения, но 
и от решения остальных участников.  Модель такого вида называется -
игрой, а учаастники конфликта - игроками. В рамках данной работы
будет рассмотрена задача из Теории некооперативных игр - 
игр, в которых игроки действуют самостоятельно, незываисимо друг 
от друга. 

Определим формально модель игры с несколькими участниками в общем виде.

Есть конечное множество $P$ игроков, которые перенумерованы
$1, 2, ..., m$. Каждый игрок из множества $P$ имеет конечное множество
чистых стратегий $S_k=\{1,2,...,n_k\}$, при этом ситуацие называется
$m$ мерный вектор
$$
\textbf{s} = (s_1, s_2, ..., s_m) \in \Motimes_{i \in P} S_i
$$

Функция выигрыша - это отображение вида вида .
\begin{equation}
	F: S_1 \times S_2 \times ... \times S_m \rightarrow \mathbb R
	\label{eq:payoff_function}
\end{equation}

которая обозначает выигрыша игрока при конкретной ситуации в игре.
Эта функция определена для каждого игрока из $P$.

\newtheorem{Def}{Определение}
\begin{Def}
	Игрой в нормалной форме называется совокупность:
	$$
		G = \lbrace P, S, F \rbrace
	$$
	где: 
 
	$P=\{1,2,...,m\}$ - множество игроков

	$S=\{S_1, S_2, ..., S_m\}$ - множество наборов чистых 
	стратегий игроков. 

	$F=\{F_1, F_2, ..., F_m\}$ - множество функция выигрыша игроков.
\end{Def}

Теперь введём фундаментальное понятие в теории игр - 
равновесие по Нэшу:

\begin{Def}
	Ситуация $\textbf{s} = (s_1, s_2, ..., s_m)$ называется
	равновесием по Нэшу игры $G = \lbrace P, S, F \rbrace$, если:
	$$
		\max \limits_{s_i \in S_i} 
		F_i(s_1^0, ..., s_{i-1}^0, s_i, s_{i+1}^0..., s_m^0)=
		F_i(s_1^0, ..., s_{i-1}^0, s_i^0, s_{i+1}^0..., s_m^0),
		\: i \in P
	$$
\end{Def}

Смысл этого определения заключается в том, что 
при ситуации в игре, которая является равновесием по Нэшу,
одному игроку индивидуально не выгодно отклоняться от своей стратегии.

До этого мы рассматривали функции выигрыша игроков, которые имели вид:
\eqref{State:opt_strat_1}, т.е. каждому игроку соответсвовало одно
значение, зависящее ситуации игры. Рассмотрим обобщение, когда 
функция выигрыша игроков имеет вид:

\begin{equation}
	F: S_1 \times S_2 \times ... \times S_m \rightarrow \mathbb R^m
	, \; m \in \mathbb{N}
	\label{eq:payoff_function}
\end{equation}

Это означает, что каждый игрок имеет некторое конечно конечно множество
критериев, в которых имеют для него значение и изменяются в зависимости
от ситуации игры. Такое обощение ближе к реальным ситуациям
в которых рассматриваются несколько значимых параметров. Для примера 
можно привести задачу выбора машины: допустим покупателю важно
чтобы машина имела большую мощность, безопасность и мало стоило
продавцу же важно, чтобы она стола как можно дороже и ещё следует
продавать машины из которые плохо продаются. Таким образом мы получили
игру, в которой игроки имеют два и тра критерия, которые важны для них
при выборе стратегии.

	Приведённые выше обощения приводят нас к другому разделу
математики, а именно: многокритериальной оптимизации. Рассмотрим
следующую задачу которая относится к этой области.

\begin{equation}
	\max\limits_{x \in X} F(x)=({f}_1(x),\ldots, {f}_n(x))
	\label{eq:mc_problem}
\end{equation}

Это задача заключается в том, что у нас есть $n$-мерная функция,
которая представляет собой множество значений критериев, зависящая
от параметров, которые принадлежат некоторому множеству.
Ньюанс заключается в том, что правило сравнения двух векторов
не определено, т.е. в общей задаче не всегда можно точно сказать,
какой из двух векторов значений функции предпочтительнее.

Введём понятие оптимальных по Парето и Слейтеру векторов задачи 
\eqref{eq:mc_problem}

\begin{Def}
	Допустимое решение $\hat{x}\in{X}$ называется 
	эффективным по Слейтеру (эффективным по Парето) для задачи

	если \textbf{не} существует $x\in{X}$ такого, что 
	$f_k(x)>f_k(\hat{x}) \; (f_k(x) \geqslant f_k(\hat{x}))$ для всех
$k=\{1,...,n\}$. Множетсво всех эффективных по Слейтеру решений называается \textit{множеством Слейтера} задачи \eqref{eq:mc_problem}.
\end{Def}



В работе рассматриваются различные способы решения модельной задачи которая представляет собой игру двух лиц с противоположными интересами и двумерной
функцией выигрыша. Для решения задачи применяется модифицированный
метод свёрток предложенный Л.С. Шепли \cite{shapley}, который как правило используется в
подобных задачах.



Для задачи \eqref{eq:mc_game} введеём следующие частные случаи:

\begin{gather*}
	S_x(y^*) \textrm{ - множество Слейтера задачи } \max\limits_{x \in X} F(x, y^*) \\
	S_y(x^*) \textrm{ - множество Слейтера задачи } \min\limits_{y \in Y} F(x^*, y)
\end{gather*}

\begin{Def}
	Решением игры \eqref{eq:mc_game} согласно \cite{blackwell} 
	является множество точек $(x^*, y^*)$ таких, что
   $x^* \in Х S_x(y^*) $ и $ y^*  \in S_y (x^*)$, 
\end{Def}

\vspace{5mm}

Для параметризации множеств Слейтера будем использовать метод свёрток.
Он заключается в том, что задача $\max\limits_{x \in X} F(x)$ заменяется параметрическим 
семейством скалярных задач $\max\limits_{x \in X} C(\{f_i\}, \lambda, x)$,
где $C$ – функция свертки частных критериев $\{f_i\}_{i=1}^m$ задачи 
\eqref{eq:mc_problem} в единый скалярный критерий, $\lambda$ – параметр свертки. 

\begin{Def}
	Линейной свёрткой с параметром $\lambda$ для функции критериев задачи
	\eqref{eq:mc_problem} называется функция:
	\begin{equation}
		L(\{f_i\}, \lambda, x)=\sum_{i=1}^{m} \lambda_i f_i \textrm{, где }
		\lambda \in \Lambda =\{\lambda_i \geq 0 | \sum_{i=1}^n \lambda_i =1 \},
		\label{eq:linear_scalarization}
	\end{equation}

	\begin{flushleft}
	свёрткой Гермейера или обратной логической свёрткой с параметром $\mu$ для 
	функции критериев задачи \eqref{eq:mc_problem} называется функция:
	\end{flushleft}	
	
	\begin{equation}
		G(\{f_i\}, \mu, x)=
		\min \limits_{i: \mu_i > 0} \frac{f_i}{\mu_i} \textrm{, где }
		\mu \in M =\{\mu_i \geq 0 | \sum_{i=1}^n \mu_i =1 \}.
		\label{eq:germeyer_scalarization}	
	\end{equation}
\end{Def}

В случае конечных $X$ и $Y$ Шепли свел \cite{shapley} описание данного множества к
семейству задач поиска значений скалярных игр с функциями выигрышей – 
ЛС частных критериев при произвольном наборе весовых коэффициентов, 
своих у каждого игрока. Применение линейной свертки в многокритериальных 
задачах обосновывается леммой Карлина

\newtheorem{Th}{Теорема}
\begin{Th}[Карлин \cite{carlin}]
	Пусть $x_0$ – эффективная точка,
    Тогда существуют неотрицательные числа $\lambda_1,…,\lambda_m$ такие, 
    что $\sum_{i=1}^m \lambda_i=1$ и $x_0$ является точкой максимума функции  
    $L(x) =\sum_{j=1}^m \lambda_j f^j(x)$ 
\end{Th}

Гермейером была предложена свертка, которая также аппроксимирует 
множество Слейтера. В работе используется ее модификация -
обратная логическая свертка, она отличается тем, что веса стоят в знаменателе.

\begin{Th}[Гермейер \cite{germeyer}]
    Пусть $x_0$ – эффективная точка, причем $f^i(x_0)>0$ для всех 
    $i=1,…,m.$
    Тогда существуют положительные числа $\lambda_1,…,\lambda_m$ такие, 
    что $\sum_{i=1}^m \lambda_i=1$ и $x_0$ является точкой максимума функции  
    $G(x) =\min \limits_{1 \leqslant j \leqslant m} \lambda_j f^j(x)$.
\end{Th}

Далее будем использовать следующие обозначения:

\begin{gather*}
	\arg \max \limits_{x \in X} f(x) = 
	\{ x^* \in X \: | \: f(x^*) = \min \limits_{x \in X} f(x)\}
	\\
	\arg \min \limits_{x \in X} f(x) = 
	\{ x^* \in X \: | \: f(x^*) = \max \limits_{x \in X} f(x)\}
\end{gather*}

Рассмотрим случай, когда \textbf{С} использует обратную логическую свертку,
с парамером $\mu$, a \textbf{П} использует линейную
свертку с параметром $\lambda$. Тогда множество оптимальных решений:

\begin{gather*}
	\begin{cases}
		x^*=\arg \max \limits_{x}   G(\{f_1, f_2\}, \mu, x, y^*) \\
		y^*=\arg \min \limits_{y}   L(\{f_1, f_2\}, \lambda, x^*, y)
	\end{cases}
\end{gather*}

Поскольку игроки используют смешанные стратегии т.е. распределения вероятностей
$\rho_x(x)$ и $\rho_y(y)$ над чистыми стратегиями $x \in X$ и $y \in Y$. 
Далее каждый игрок осредняет свою функцию выигрыша по стратегиям противника

$$ \overline G(\{f_1, f_2\}, \mu, q, p) = 
\iint \limits_{PQ} G(\{f_1, f_2\}, \mu, q, p) \rho_x(x) \rho_y(y)dxdy$$
$$ \overline L(\{f_1, f_2\}, \lambda, q, p) = 
\iint \limits_{PQ} L(\{f_1, f_2\}, \lambda, q, p) \rho_x(x) \rho_y(y)dxdy$$

\begin{Def}	\label{def:optimal_strategy}
	Пара стратегий $(p^0, q^0)$ называется оптимальными, если для некоторых 
	$\lambda$, $\mu$ верно:
	\begin{equation}
		\begin{cases} 
		p^0(q^0, \lambda) = \argmin\limits_{p \in P} \overline L(p, q^0, \lambda) \\ 
		q^0(p^0, \mu) = \argmin\limits_{q \in Q} \overline G(p^0, q, \mu) \\
	\end{cases}
	\end{equation}
\end{Def}

Мы будем рассматривать конечную игру \textbf{C --- П}, полученную из исходной 
заменой множества $X=[0,1]$ конечным множеством точек

$$X^T = \{
	0, \; \frac{1}{T}, \; \frac{2}{T}, \; \ldots, \; \frac{T-1}{T}, \; 1
\}, \;\; T \in \mathbb{N}$$

В работе исследуются случаи $T=1$: $X^1=\{0, 1\}$, 
и $T=2$: $X^2=\{0, \frac{1}{2} ,1\}$