\section{Множество оптимальных стратегий}

Для поиска оптимальных стратегий сначала необходимо установить значения 
$p^*(q, \lambda) = \arg \max \limits_{p \in P} \overline L(p, q, \lambda)$ и 
$q^*(p, \mu) = \arg \min \limits_{q \in Q} \overline G(p, q, \mu)$.

Из статьи \cite{novikova} следует, что:
\begin{equation}
	p^*(q, \lambda)=
	\arg \max \limits_{p \in P} \overline L(p, q, \lambda) =
	\begin{cases}
		0, & q > 1 - \lambda \\
		1, & q < 1 - \lambda \\
		[0,1], & q = 1 - \lambda
	\end{cases}
\end{equation}

Из пункта 3 следует, что
\begin{equation}
	q^*(p, \mu) = \arg \min \limits_{q \in Q} \overline G(p, q, \mu) =
	\begin{cases}
		\dfrac{\mu}{2 - \mu}, & p \geqslant 1 - \mu \\
		\dfrac{2\mu}{1 + \mu}, & p \leqslant 1 -\mu \\
	\end{cases}
\end{equation}


\newtheorem{State}{Утверждение}
\begin{State}
	Любая пара $(p^{*}, q^{*}) \in [0, 1]^{2}$ является оптимальной, т.е.  
	$\forall \; (p^{*}, q^{*}) \in [0, 1]^{2} \; \exists \; 
	(\mu, \lambda) \in M \times \Lambda$: 
	верно \eqref{def:optimal_strategy}.
\end{State}

\begin{flushleft}
	\textbf{Доказательство}

Зафиксируем некоторую пару  $(p^{*}, q^{*}) \in [0, 1]^{2}$ и найдём такие 
$\hat{\mu}(p^{*}, q^{*})$ и $\hat{\lambda}(p^{*}, q^{*})$ 
что верно \eqref{def:optimal_strategy} т.е. что пара стратегий является оптимальной.

\textbf{(1)} Покажем оптимальность $p^{*}$:

Возьмём $\hat \lambda := 1 - q^* \; \Rightarrow$  поскольку 
$\arg \min\limits_{p \in P} \overline L(p, q^*, \hat \lambda) = [0,1]$ 
при $ q^* = 1 - \hat \lambda$, 
то и $p^* \in \arg \min \limits_{p \in P} \overline L(p, q^*, \hat \lambda)$


\textbf{(2)} Покажем оптимальность $q^{*}$. По имеющемуся $q^*$ 
решим уравнения $f_1(\mu_1)=q^*$ и $f_2(\mu_2)=q^*$, где
$f_1=\dfrac{\mu}{2 - \mu}$ и $f_2=\dfrac{2 \mu}{1 + \mu}$:

\begin{gather*}
	q^*=\dfrac{2 \mu_2}{1 + \mu_2} \Rightarrow \mu_2 = \frac{q^*}{2 - q^*} \\
	q^*=\dfrac{\mu_1}{2 - \mu_1} \Rightarrow \mu_1 = \frac{2 q^*}{1 + q^*}
\end{gather*} 


Заметим, что при $q^* \in [0, 1]$ верно $\mu_2 \leqslant \mu_1$.
В таком случае поскольку $1 - p^* \in [0, 1]$, то при фиксированных переменных 
$(p^*, q^*)$ будет реализован один и только один из 3-х вариантов:

\textbf{(a)}
$1 - p^* \leqslant \mu_2$, т.е.  $1 - p^* \leqslant \dfrac{q^*}{2 - q^*}$.
В этом случае $\hat \mu := \mu_1 = \dfrac{2 q^*}{1 + q^*}$.

Действительно, при таком выборе $\hat{\mu}$ имеем: 

\begin{gather*}
	1 - p^* \leqslant \mu_2 \leqslant \mu_1 = \hat \mu \Rightarrow
	p^* + \hat \mu - 1 \geqslant 0 \Rightarrow 
	\\	
	\Rightarrow \arg \max \limits_{q \in Q} \overline G(p^*, q, \hat \mu) =
	\dfrac{\hat{\mu}}{2-\hat{\mu}} =
	\dfrac{\dfrac{2 q^*}{1 + q^*}}{2 - \dfrac{2 q^*}{1 + q^*}} =
	q^* \Rightarrow
\end{gather*} 

$$
	\textrm{при}
	\begin{cases}
		\hat \mu = \dfrac{2 q^*}{1 + q^*} \\
		\hat \lambda = 1 - q^* \\
		1 - p^* \leq \frac{q^*}{2 - q^*}
	\end{cases}
	(p^*, q^*) - \textrm{оптимальная пара}
$$

\textbf{(b)}
$\mu_2 < 1-p^* \leqslant \mu_1$, т.е.  
$\dfrac{q^*}{2 - q^{*}} < 1 - p^* \leqslant \dfrac{2 q^*}{1 + q^*}$.
В этом случае $\hat \mu := \mu_1 = \dfrac{2 q^*}{1 + q^*}$.
Действительно, при таком выборе $\hat \mu$ имеем:

\begin{gather*}
	1 - p^* \leq \mu_1 = \hat \mu 
	\; \Rightarrow \;
	p^* + \hat \mu - 1 \geqslant 0 \Rightarrow 
	\\
	\Rightarrow  \arg \max \limits_{q \in Q} \overline G(p^*, q, \hat \mu) =
	\dfrac{\hat \mu}{2 - \hat \mu} =
	\dfrac{\dfrac{2 q^*}{1 + q^*}}{2 - \dfrac{2 q^*}{1 + q^*}}=
	q^* \Rightarrow
\end{gather*} 

$$
\textrm{при}
\begin{cases}
	\hat \mu = \dfrac{2 q^*}{1 + q^*} \\
	\hat \lambda = 1 - q^* \\
	\dfrac{q^*}{2 - q^*} < 1 - p^* \leqslant \dfrac{2 q^*}{1 + q^*}
\end{cases}
(p^*, q^*) - \textrm{оптимальная пара}
$$

\textbf{(c)} 
$\mu_1 \leqslant 1-p^*$ т.е. $\dfrac{2q^*}{1 + q^*} < 1 - p^*$.
В этом случае $\hat \mu := \mu_2 = \dfrac{q^*}{2 - q^*}$ 

действительно, при таком выборе $\hat \mu$ имеем 

\begin{gather*}
	1 - p^* \geqslant \mu_2 = \hat \mu \Rightarrow
	p^*+\mu-1 \leqslant 0 \Rightarrow 
	\\	
	\Rightarrow \arg \max \limits_{q \in Q} \overline G(p^*, q, \hat \mu) =
	\dfrac{2 \hat \mu}{1 + \hat \mu} =
	\dfrac{2 \dfrac{q^*}{2 - q^*}}{1 + \dfrac{q^*}{2 - q^*}} =
	q^* \Rightarrow
\end{gather*}
 
$$
\textrm{при}
\begin{cases}
	\hat\mu=\dfrac{q^{*}}{2-q^{*}} \\
	\hat\lambda = 1 - q^{*} \\
	\dfrac{2q^{*}}{1+q^{*}} < 1-p^{*} \\
\end{cases}
(p^*, q^*) - \textrm{оптимальная пара}
$$
 
\end{flushleft}


Таким образом при имеющихся $(p^{*},q^{*})$ сначала надо определить в какой из вариантов 1-3 пара попадает, затем
по соответсвующим равенствам найти $\hat\mu$ и $\hat\lambda$.

\textbf{Утверждение доказано.}
