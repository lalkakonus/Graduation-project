
\hspace{5mm}

Теперь для каждой пары параметров $(\mu, \lambda)$ найдём множество
соответствующих оптимальных пар 
$(p^0(\mu, \lambda), q^0(\mu, \lambda)) \in P \times Q$. 
Рассмотрим все возможные сочетания значений для $p^*$ и $q^*$ в системах 
\eqref{eq:argmin_L_1} \eqref{eq:argmax_G_1}, что даст нам 6 следующих систем:

\hspace{3mm}

	\textit{Учтём, что переменные $p, q, \mu$ и $\lambda$ определены на отрезке $[0, 1]$.}
	
%---------------------------1------------------------
\textbf{(1)}
$$
	\begin{cases}
		p^* = 0 \\
		q^* = \dfrac{\mu}{2 - \mu} \\
		q^* > 1 - \lambda \\
		p^* + \mu - 1 \geqslant 0 \\
	\end{cases}
	\quad \sim \quad
	\begin{cases}
		p^* = 0 \\
		q^* = \dfrac{\mu}{2 - \mu} \\
		\dfrac{\mu}{2 - \mu} > 1 - \lambda \\
		\mu \geqslant 1 \quad \Rightarrow \quad \mu = 1 \\
	\end{cases}
	\quad \sim \quad
	\begin{cases}
		p^* = 0 \\
		q^* = 1 \\
		\lambda > 0 \\
		\mu = 1
	\end{cases}
$$

При $\mu = 1$  и $\lambda \in (0,1]$ имеем следующие оптимальные пары:
$$
	(p^0, q^0) \in (0, 1).
$$

%---------------------------2------------------------
\textbf{(2)}
$$
	\begin{cases}
		p^* = 0 \\
		q^* = \dfrac{2\mu}{1+\mu} \\
		q^* > 1 - \lambda \\
		p^* + \mu - 1 \leqslant 0 \\
	\end{cases}
	\quad \sim \quad
	\begin{cases}
		p^* = 0 \\
		q^* = \dfrac{2\mu}{1+\mu} \\
		\dfrac{2\mu}{1+\mu} > 1 - \lambda \\
		\mu \leqslant 1
	\end{cases}
	\quad \sim \quad
	\begin{cases}
		p^* = 0 \\
		q^* = \dfrac{2\mu}{1+\mu} \\
		\lambda > \dfrac{1-\mu}{1+\mu} \\
		\mu \leqslant 1
	\end{cases}
$$

При $\mu \in [0, 1]$ и $\lambda \in (\frac{1-\mu}{1+\mu}, 1]$
имеем следующие оптимальные пары:
$$
	(p^0, q^0) \in (0, \dfrac{2\mu}{1 + \mu}).
$$

%---------------------------3------------------------
\textbf{(3)}
$$
	\begin{cases}
		p^* = 1 \\
		q^* = \dfrac{\mu}{2-\mu} \\
		q^* < 1 - \lambda \\
		p^* + \mu - 1 \geqslant 0 \\
	\end{cases}
	\quad \sim \quad
	\begin{cases}
		p^* = 1 \\
		q^* = \dfrac{\mu}{2-\mu} \\
		\dfrac{\mu}{2-\mu} < 1 - \lambda \\
		\mu \geqslant 0
	\end{cases}
	\quad \sim \quad
	\begin{cases}
		p^* = 1 \\
		q^* = \dfrac{\mu}{2-\mu} \\
		\lambda < 2\dfrac{1-\mu}{2-\mu} \\
		\mu \geqslant 0
	\end{cases}
$$

При $\mu \in [0, 1]$ и $\lambda \in [0, 2\frac{1-\mu}{2-\mu})$
имеем следующие оптимальные пары:
$$
	(p^0, q^0) \in (1, \frac{\mu}{2 - \mu}).
$$

%---------------------------4------------------------
\textbf{(4)}
$$
	\begin{cases}
		p^* = 1 \\
		q^* = \dfrac{2\mu}{1+\mu} \\
		q^* < 1 - \lambda \\
		p^* + \mu - 1 \leqslant 0 \\
	\end{cases}
	\quad \sim \quad
	\begin{cases}
		p^* = 1 \\
		q^* = \dfrac{2\mu}{1+\mu} \\
		\dfrac{2\mu}{1+\mu} < 1 - \lambda \\
		\mu \leqslant 0 \quad \Rightarrow \quad \mu = 0 \\
	\end{cases}
	\quad \sim \quad
	\begin{cases}
		p^* = 1 \\
		q^* = 0 \\
		\lambda < 1 \\
		\mu = 0
	\end{cases}
$$

При $\mu=0$ и $\lambda \in [0, 1)$ имеем следующие оптимальные пары:
$$
	(p^0, q^0) \in (1, 0).
$$

%---------------------------5------------------------
\textbf{(5)}
$$
	\begin{cases}
		p^* \in [0, 1] \\
		q^* = \dfrac{\mu}{2-\mu} \\
		q^* = 1 - \lambda \\
		p^* + \mu - 1 \geqslant 0 \\
	\end{cases}
	\quad \sim \quad
	\begin{cases}
		p^* \in [0, 1] \\
		q^* = \dfrac{\mu}{2-\mu} \\
		\dfrac{\mu}{2-\mu} = 1 - \lambda \\
		p^* \geqslant 1 - \mu \\
	\end{cases}
	\quad \sim \quad
	\begin{cases}
		p^* \in [1 - \mu, 1] \\
		q^* = \dfrac{\mu}{2-\mu} \\
		\lambda = 2\dfrac{1-\mu}{2-\mu} \\
	\end{cases}
$$

При $\mu \in [0, 1]$ и $\lambda = 2\dfrac{1 - \mu}{2 - \mu}$ 
имеем следующие оптимальные пары:
$$
	(p^0, q^0) \in [1 - \mu, 1] \times  \{ \frac{\mu}{2 - \mu}\}.
$$

%---------------------------6------------------------
\textbf{(6)}
$$
	\begin{cases}
		p^* \in [0, 1] \\
		q^*= \dfrac{2\mu}{1+\mu} \\
		q^* = 1 - \lambda \\
		p^* + \mu - 1 \leqslant 0 \\
	\end{cases}
	\quad \sim \quad
	\begin{cases}
		p^* \in [0, 1] \\
		q^* = \dfrac{2\mu}{1+\mu} \\
		\dfrac{2\mu}{1+\mu} = 1 - \lambda \\
		p^* \leqslant 1 - \mu 
	\end{cases}
	\quad \sim \quad
	\begin{cases}
		p^* \in [0, 1 - \mu] \\
		q^* = \dfrac{2\mu}{1+\mu} \\
		\lambda = \dfrac{1-\mu}{1+\mu} \\
	\end{cases}
$$

При $\mu \in [0, 1]$ и $\lambda = \dfrac{1-\mu}{1+\mu}$
имеем следующие оптимальные пары:
$$
	(p^0, q^0) \in [0, 1 - \mu] \times \{\frac{2\mu}{1 + \mu}\}.
$$
Итого получаем:

$$
	\mathbb{O}_1 =
	\begin{cases}
		(0, \, 1), & \mu = 1, \: \lambda \in (0,1] 
		\\
		(0, \, \dfrac{2\mu}{1 + \mu}), & 
		\mu \in [0, 1], \: \lambda \in (\frac{1-\mu}{1+\mu}, 1]
		\\
		(1, \, \dfrac{\mu}{2 - \mu}), & 
		\mu \in [0, 1], \: \lambda \in [0, 2\frac{1-\mu}{2-\mu})
		\\
		(1, \, 0), & \mu=0, \: \lambda \in [0, 1)
		\\
		[1 - \mu, 1] \times  \Big\{ \dfrac{\mu}{2 - \mu}\Big\}, &
		\mu \in [0, 1], \: \lambda = 2\dfrac{1 - \mu}{2 - \mu}	
		\\
		[0, 1 - \mu] \times \Big\{\dfrac{2\mu}{1 + \mu}\Big\}, &
		\mu \in [0, 1], \: \lambda = \dfrac{1-\mu}{1+\mu}
		\\
	\end{cases}
$$

Некоторые условия оптимальных пар пересекаются, поэтому 
произведём агрегацию системы по условиям таким образом, чтобы
множества $(\mu, \lambda)$, которые стоят в правой части,
имели между собой пустое пересечение.
$$
	\mathbb{O}_1 =
	\begin{cases}
		(1, \, 0), & \mu = 0, \lambda \in [0, 1) 
		\\
		[0, 1] \times \{0\}, & 
		\mu=0, \lambda =1
		\\
		[0, 1] \times \{1\}, &
		\mu = 1, \lambda = 0
		\\
		\{1\} \times \{\dfrac{\mu}{2-\mu}\}, &
		\mu \in (0,1), \lambda \in (0, \dfrac{1 - \mu}{1 + \mu}]
		\\
		\{1\} \times \{\frac{\mu}{2-\mu}\} \cup
		[0,1-\mu] \times \{\frac{2\mu}{1+\mu}\} &
		\mu \in (0,1), \lambda = \dfrac{1-\mu}{1+\mu}
		\\
		(0, \dfrac{2\mu}{1 + \mu}) \cup
		(1, \dfrac{\mu}{2 - \mu}) &
		\mu \in [(, 1), \lambda \in 
		[0, 2\dfrac{1 - \mu}{2 - \mu}] \cap (\dfrac{1 - \mu}{1 + \mu}, 	1]
		\\
		(0, \dfrac{2\mu}{1+\mu}) \cup
		[1 - \mu, 1] \times \{\dfrac{\mu}{2 - \mu}\} &
		\mu \in (0, 1), \lambda = 2\dfrac{1 - \mu}{2 - \mu}
		\\
		(0, \dfrac{2\mu}{1 + \mu}) &
		\mu \in (0, 1), \lambda \in (2\dfrac{1 - \mu}{2 - \mu}, 1]
		\\
		(0, 1) & \mu = 1, \lambda \in (0, 1] 	
	\end{cases}
$$

Каждое из множеств в условиях системы изображено на отдельном графике
на рис. \ref{fig:lambda_mu_set} снизу. Были рассмотрены все точки
множества $(\mu, \lambda) \in [0,1]^2$.

\begin{figure}[H]
	\centering
	\begin{subfigure}[b]{0.3 \textwidth}
		\includegraphics[width=\linewidth]{part_2/graf_3_1}
		\caption{$\mu = 0, \: \lambda \in [0, 1)$}
	\end{subfigure}
	\begin{subfigure}[b]{0.3 \textwidth}
		\includegraphics[width=\linewidth]{part_2/graf_3_2}
		\caption{
			$
			\left[
			\begin{array}{c}
     			\mu=0, \: \lambda = 1 \\
     			\mu=1, \: \lambda = 0
  			\end{array}
			\right.
			$
		}
	\end{subfigure}
	\begin{subfigure}[b]{0.3 \textwidth}	
		\includegraphics[width=\linewidth]{part_2/graf_3_3}
		\caption{
			$
				\mu \in (0,1), \:
				\lambda \in (0, \frac{1 - \mu}{1 + \mu}]
			$		
		}
	\end{subfigure}
	\newline	
	\centering
	\begin{subfigure}[b]{0.3 \textwidth}
		\includegraphics[width=\linewidth]{part_2/graf_3_4}
		\caption{
			$\mu \in (0,1), \lambda = \dfrac{1-\mu}{1+\mu}$		
		}
	\end{subfigure}
	\begin{subfigure}[b]{0.3 \textwidth}
		\includegraphics[width=\linewidth]{part_2/graf_3_5}
		\caption{
			$\mu \in [(0, 1), \newline 
			\lambda \in 
			[0, 2\frac{1 - \mu}{2 - \mu}] \cap 
			(\frac{1 - \mu}{1 + \mu}, 1]$
		}
	\end{subfigure}
	\begin{subfigure}[b]{0.3 \textwidth}	
		\includegraphics[width=\linewidth]{part_2/graf_3_6}
		\caption{
			$\mu \in (0, 1), \lambda = 2\dfrac{1 - \mu}{2 - \mu}$: 
		}
	\end{subfigure}
	\newline
	\begin{subfigure}[b]{0.3 \textwidth}
		\includegraphics[width=\linewidth]{part_2/graf_3_7}
		\caption{
			$\mu \in (0, 1), 
			\lambda \in (2\frac{1 - \mu}{2 - \mu}, 1]$		
		}
	\end{subfigure}
	\begin{subfigure}[b]{0.3 \textwidth}
		\includegraphics[width=\linewidth]{part_2/graf_3_8}
		\caption{
			$\mu = 1, \lambda \in (0, 1] $		
		}
	\end{subfigure}
	\caption{}
	\label{fig:lambda_mu_set}
\end{figure}