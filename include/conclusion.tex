\section{Заключение}

На примере двухкритериальной игры двух лиц была изучена возможность 
применения линейной свёртки и обратной логической свёртки (кторая 
является модификацией свёртки Гермейера). Рассмотрены два варианта 
дискретизации непрерывной модельной задачи и получены следующие результаты:
\begin{enumerate}
	\item \textbf{Оптимальные стратегии}. В случае Т=1 оптимальные
	стратегии оказались неизирательными, поскольку любая
	допустимая стратения оказалась оптимальной. В случае Т=2
	для игрока \textbf{П} оптимальна любая допустимая сттратегия, а
	для игрока \textbf{С} множество оптимальных стратегий состоит из
	двух отрезков.  
 
	\item \textbf{Множества значений в критериальном пространстве}. 
	Выпуклая оболочка множества значений в критериальном пространстве 
	при оптимальных стратегиях в случае Т=2 содержит в себе это
	множество для Т=1. Более при увеличении степени аппроксимации,
	это множество всё точнее приближается к непрерывному случаю.
	
\end{enumerate}
