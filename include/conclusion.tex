\section{Заключение}

Была исследована и решена модельная двухкритериальная игры двух лиц.
На её примере было изучено применение различных свёрток игроками,
и проанализирована зависимость выигрыша от подхода к решению задачи.
Были рассмотрены два варианта дискретизации непрерывной модельной задачи и получены следующие результаты:
\begin{enumerate}
	\item \textbf{Оптимальные стратегии}. В случае Т=1 оптимальные
	стратегии оказались неизбирательными, поскольку любая
	допустимая стратегия оказалась оптимальной. В случае Т=2
	для игрока \textbf{П} оптимальна любая допустимая стратегия, а
	для игрока \textbf{С} множество оптимальных стратегий состоит из
	двух отрезков.  
 
	\item \textbf{Множества значений в критериальном пространстве}. 
	Выпуклая оболочка множества значений в критериальном пространстве 
	при оптимальных стратегиях в случае Т=2 содержит в себе это
	множество для Т=1. Более того, при увеличении степени аппроксимации,
	это множество всё точнее приближается к непрерывному случаю.
\end{enumerate}

\begin{flushleft}
	Работа была представлена на конференции "Ломоносовские чтения 2019"
	\cite{kononov}.
\end{flushleft}