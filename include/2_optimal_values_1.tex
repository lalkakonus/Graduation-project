\section{Решение игры. \\Параметр дискретизации Т=2}

\subsection{Оптимальная стратегия преподавателя}

Игрок \textbf{П} стремится минимизировать вектор-функцию выигрыша  \eqref{eq:player_criterion}: 
$$
	F(x, y) = \Big \langle
		\frac{y\sqrt{x}}{2},\frac{\sqrt{1-x}}{y}
	\Big \rangle
$$	
Он использует смешанную стратегию, т.е. его стратегией является
распределение $p=(p_0,p_1) \in P$ над множеством чистых стратегий $Y=\{1,2\}$.
Его вектор-функция выигрыша \eqref{eq:player_criterion} приобретает вид:
\begin{multline*}
	F_\textrm{П}(p,x)=
	\big \langle 
		(1-p)\frac{1 \cdot \sqrt{x}}{2} + p \frac{2 \cdot \sqrt{x}}{2}; \;
		(1-p)\frac{\sqrt{1-x}}{1}+p\frac{\sqrt{1-x}}{2} 
	\big \rangle= \\
	=\frac{1}{2}
	\big \langle
		(p+1)\sqrt{x}; \;
		(2-p)\sqrt{1-x}
	\big \rangle.
\end{multline*}
Затем использует \textbf{ЛС} \eqref{eq:linear_scalarization}:
$$
	L(p,x,\lambda)=
	\frac{1}{2}
	\big(
		\lambda(p+1)\sqrt{x} + (1-\lambda)(2-p)\sqrt{1-x}
	\big)
$$
Далее осредняем 
функцию $L(p,x,\lambda)$ по стратегиям противника $x \in X=\{0,\frac{1}{2},1\}$ 
c вероятностями $q=(q_0,q_1,q_2)$:
\begin{multline*} 
	\overline{L}(p,q,\lambda)=
	\frac{1}{2} 
	\Big(
		q_0(1-\lambda)(2-p)\sqrt{1}+
		q_1 \big (\lambda(p+1)\frac{1}{\sqrt{2}} + \\
		+(1-\lambda)(2-p)\frac{1}{\sqrt{2}} \big )+
		q_2\lambda(p+1)\sqrt{1}
	\Big)=
	\\
	=\frac{1}{2\sqrt{2}}
	\Big (
		\big (\lambda(q_1+\sqrt{2}q_2)-(1-\lambda)(q_1+\sqrt{2}q_0) \big)p+
		\big (\lambda(q_1+\sqrt{2}q_2)+2(1-\lambda)(q_1+\sqrt{2}q_0) \big)
	\Big).
\end{multline*}
Функция является линейной по переменной $p$:
$$
	\overline{L}(p,q,\lambda)=k(\lambda,q)p+b(\lambda,q),
$$
где:
\begin{gather*}
	k(q, \lambda) = \frac{1}{2\sqrt{2}}
	\big (\lambda(q_1+\sqrt{2}q_2)-(1-\lambda)(q_1+\sqrt{2}q_0) \big)	,
	\\
	b(q, \lambda) = \frac{1}{2\sqrt{2}}
	\big (\lambda(q_1+\sqrt{2}q_2)+2(1-\lambda)(q_1+\sqrt{2}q_0) \big).
\end{gather*}
Наша задача -- найти 
$p^*(q, \lambda)=\arg \min \limits_{p \in [0, 1]} \overline{L}(p,q,\lambda)$.
Поскольку функция $\overline{L}(p,q,\lambda)$ линейна по переменной 
$p$, следовательно:
$$
	p^*(q, \lambda) =
	\arg \min \limits_{p \in [0, 1]} \overline{L}(p,q,\lambda) =
	\begin{cases}
		0, & k(\lambda,q)>0 \\
		1, & k(\lambda,q)<0 \\
		[0,1], & k(\lambda,q)=0
	\end{cases}	
$$
Рассмотрим функцию $k(q, \lambda)$:
$$
	k(q, \lambda)=\frac{1}{2\sqrt{2}}
	\Big(
		\lambda \big (2+(\sqrt{2}-2)(q_0-q_2) \big) -
		\big (1 - q_2 + (\sqrt{2} - 1)q_0 \big)
	\Big)
$$
Нас интересует знак это функции при различных значениях аргументов.
Введём обозначения:
$$
	\ell(q) = \frac{1 - q_2 + (\sqrt{2} - 1)q_0}{2+(\sqrt{2}-2)(q_0-q_2)}
$$
Напомним, что множество $Q$ имеет следующий вид:
$$
	Q = \{
		(q_0,q_2) \in \mathbb{R}^2_{+} \; | \;
		q_0 + q_2 \leqslant 1
	\}	
$$	
Поскольку для $q \in Q$ верно: 
\begin{gather*}
	2+(\sqrt{2}-2)(q_0-q_2) > 0 
	\\
	1 - q_2 + (\sqrt{2} - 1)q_0 \geqslant 0 
	\\
	q_0 + (\sqrt{2} - 3) q_2 - 1 \leqslant 0
\end{gather*}
Следовательно:
$$
	k(q, \lambda) \vee 0 \Leftrightarrow 
	\lambda \vee \ell(q) \textrm{ где } \vee 
	\textrm{ это один из знаков } >,<,=.
$$
	
Более того верно что 
$\forall \; q \in Q: \; 0 \leqslant \ell(q) \leqslant 1$. 
Проиллюстрируем это на рисунке \ref{fig:l}. 
В плоскости $q=(q_0,q_2)$ изображены 
прямые $\ell_1(q)$ и $\ell_2(q)$ такие, что:
\begin{gather*}
	\ell_1(q): \; \ell(q)=0 
	\\
	\ell_2(q): \; \ell(q)=1
\end{gather*}
Зелёным цветом изображена область в которой
$0 \leqslant \ell(q) \leqslant 1$.
Видно, что квадрат $q = [0,1]^2$ а следовательно и
множество $Q$ полностью принадлежит этой области.
\begin{figure}[H]
	\centering
  	\includegraphics[scale=0.6]{part_3/graf_3_1}
  	\caption{}
  	\label{fig:l}
\end{figure}

Следовательно $\forall \; q=(q_0, q_2) \in Q \;
\exists \; \lambda \in [0,1]: \; k(\lambda,q)=0$.
\begin{equation}
	\label{eq:p*}
	p^*(\lambda,q)=
	\arg \min \limits_{p \in [0, 1]} \overline{L}(p,q,\lambda)=
	\begin{cases}
		0, & \lambda \in \big(\ell (q), 1\big] \\
		1, & \lambda \in \big[0, \ell(q) \big) \\
		[0,1], & \lambda=\ell(q)
	\end{cases},
\end{equation}

где $\ell(q)=\dfrac{1 - q_2 + (\sqrt{2} - 1)q_0}{2+(\sqrt{2}-2)(q_0-q_2)}$.