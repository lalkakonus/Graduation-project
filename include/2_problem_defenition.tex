\section{Постановка задачи (Парметр дискретизации Т=2)}

Теперь рассмотри случай с параметром дискретизации $T=2$. Тогда множество 
$X^T$ принимает вид $X^2=\{0, \frac{1}{2}, 1\}$.
Игроки используют смешанные стратегии т.е. вероятностноые распределения над 
своими чистыми стратегиями.
Чистыми стратегиями игроков \textbf{С} и \textbf{П} являются 
$X^2=\{0, \frac{1}{2}, 1\}$ и $Y=\{1,2\}$ соответсвенно. 
Эти множества дискретны, поэтому вероятностные
распределения над ними задаются в виде векторов: 
\begin{gather*}
	(q_1, q_2, q_3) \in Q_3 = \{
		(q_1, q_2, q_3) \in \mathbb{R}_+^3 \; | \; q_1 + q_2 + q_3 = 1)
	\},
	\\
	(p_1, p_2) \in P_2 = \{
		(p_1, p_2) \in \mathbb{R}_+^2 \; | \; p_1 + p_2 = 1)
	\}
\end{gather*}

Смешанные стратеги игроков \textbf{С} и \textbf{П} будем обозначать
$q=(q_0,q_1) \in Q_2$ и $p=(p_0, p_1, p_2) \in P_3$ соответсвенно, причём:
\begin{equation}
\begin{tabu} to 0.9 \textwidth {X[c] X[c]}
	$P(X=0)=q_0$ & $P(Y=1)=p_0$ \\
	$P(X=\frac{1}{2})=q_1$ & $P(Y=2)=p_1$ \\
	$P(X=1)=q_2$ & \
	\\
	\end{tabu}	
\label{eq:probability_1}
\end{equation}
Введём обозначения $p := p_1$, тогда $q_1 = 1 - q_0 - q_2$ и $p_0 = 1 - p$. 
Тогда $q=(q_0, q_2) \in Q$:
$$
	Q = \{
		(q_0,q_2) \in \mathbb{R}^2_{+} \; | \;
		q_0 + q_2 \leqslant 1
	\}	
$$
И рассматривать игру мы будем на этом множестве.

Сначала определим функции выигрыша $\overline G(p, q, \mu)$
и $\overline L(p, q, \lambda)$, а замем найдём их
найти точки максимума и минимума при фиксированных параметрах
свёрток:
\begin{gather*}
	q^*(p, \mu) = \arg \max \limits_{q \in Q} \overline G(p, q, \mu)
	\\	
	p^*(q, \lambda) = \arg \min \limits_{p \in P} \overline L(p, q, \lambda).
\end{gather*}

Затем найдём множесво оптимальных стратегий \eqref{def:optimal_strategy}.
