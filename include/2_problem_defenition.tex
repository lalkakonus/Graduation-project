\section{Постановка задачи (значение параметра T=1)}

\subsection{Игровая модель}

\qquad Рассматриваются два игрока - Студент, далее обозначается \textbf{С},
и Преподаватель, далее обозначается \textbf{П}, которые имеют противоположные интересы.
Критерий интересов составляют две велечины, первая из которых является эффективностью
работы \textbf{С} в научной сфере, а второй его эффективностью на подработке.

\textbf{С} выбирает долю $x$ рабочего времени, которую он тратит на подготовку
диплома, оставшееся рабочее время $1-x$ он тратит на подработку. Считается, что производительность \textbf{С} при любых занятий падает с увеличением 
отводимого на них времени, эффективность труда \textbf{С} зададим функцией $\sqrt{x}$
и $\sqrt{1-x}$ соответственно. \textbf{С} может распределять свое время между двумя 
видами деятельности, т.е. имеет множетсво стратегий $x\in X = \{0, 1\}$, 
причём он может использовать смешанные стратегии.

\textbf{П} выбирает - отностится к \textbf{С} требовательно, способствую 
написанию диплома и мешая подработке или же не обращать на него внимания не 
мешая подработке и не помогая с дипломом. \textbf{П} имеет множество стратегий 
$y \in Y=\{1, 2\}$, причём тоже может использовать смешаные стратегии.

\vspace{5mm}
Получаем следующий функциональный критерий:

\begin{equation}
	F(x, y)=
	\big(f_1(x,y), f_2(x,y)\big) =
	\Big(
		\frac{y\sqrt{x}}2,
		\frac{\sqrt{1-x}}y
	\Big)
	\label{eq:player_criterion}
\end{equation}

\qquad \textbf{П} стремится минимизировать (выборая $y \in Y = \{1,2\}$)
критерий $F(x, y)$, а игрок \textbf{С} - максимизировать
 (выбирая $x \in X=[0,1]$).

\vspace{5mm} 
Задачу можно представить в виде многокритериальную игру двух лиц 
с противоположными интересамив 

\begin{equation}
	\bigg \langle F(x,y), X, Y \bigg \rangle, \; y \in Y=\{1, 2\}, \; x \in {X}=[0, 1]
	\label{eq:mc_game}
\end{equation}

\subsection{Постановка зададчи}

Сначала рассмотри случай с параметром $T=1$. Тогда множество $X=\{0, 1\}$.
Игроки используют смешанные стратегии т.е. распределение над своими чистыми стратегиями.
Чистыми стратегиями игроков \textbf{С} и \textbf{П} являются $X=\{0, 1\}$ и $Y=\{1,2\}$ 
соответсвенно. Эти множества дискретны, поэтому распределения задаются в виде векторов 
вида: 

$$(p_1, p_2) \in P = \{(p_1, p_2) \in R_+^2 \; | \; p_1 + p_2 = 1)$$

Множество $Q=P$, введено для наглядности. 
Смешанные стратеги игроков \textbf{С} и \textbf{П} будем обозначать
$q=(q_0,q_1) \in Q$ и $p=(p_0,p_1) \in P$ соответсвенно, причём:

\begin{equation}
\begin{tabu} to 0.9 \textwidth {X[c] X[c]}
	$P(X=0)=q_0$ & $P(Y=1)=p_0$ \\
	$P(X=1)=q_1$ & $P(Y=2)=p_1$ \
	\\
	\end{tabu}	
\label{eq:probability_1}
\end{equation}

Введём обозначения $q := q_1$ и $p := p_1$, тогда $q_0 = 1-q$ и $p_0 = 1 - p$. 
Игрок \textbf{С} использует смешаную стратегию  $(q_0,q_1)$, тогда
его векторный критерий \eqref{eq:player_criterion} приобретает вид: 

$$
	F_\textrm{C}(q,y)=
	\big \langle
		q_0\frac{y\sqrt{0}}{2} + 
		q_1\frac{y\sqrt{1}}{2};
		q_0\frac{\sqrt{1}}{y} +
		q_1\frac{\sqrt{0}}{y}
	\big \rangle 
	= 	
	\big \langle
		\frac{q_1y}{2};
		\frac{q_0}{y}
	\big \rangle 
$$

Игрок \textbf{П} использует смешаную стратегию  $(p_0,p_1)$, тогда
его векторный критерий \eqref{eq:player_criterion} приобретает вид: 

\begin{gather*}
	F_\textrm{П}(x,p)=
	\big \langle 
		(1-p)\frac{1 \cdot \sqrt{x}}{2} + p \frac{2 \cdot \sqrt{x}}{2}; \;
		(1-p)\frac{\sqrt{1-x}}{1}+p\frac{\sqrt{1-x}}{2} 
	\big \rangle=
	\\
	=\frac{1}{2}
	\big \langle
		(p+1)\sqrt{x}; \;
		(2-p)\sqrt{1-x}
	\big \rangle
\end{gather*}
	
\vspace{5mm}

Далее игрок \textbf{С} использует \textit{обратную логическую свёртку}
\eqref{eq:germeyer_scalarization}:

$$
	G(y, q, \mu) = 
	\min \limits_{i: \mu_i > 0} \{\frac{q_1y}{2\mu_0};\frac{q_0}{y\mu_1}\},
$$

a игрок \textbf{П} использует \textit{линейную свёртку}
\eqref{eq:linear_scalarization}:

$$
	L(p, x, \lambda) = 
	\lambda_0 (p+1)\sqrt{x} + \lambda_1 (2-p)\sqrt{1-x}
$$

После чего игрок \textbf{С} осредняет свёртку критерия по стратегиям игрока \textbf{П},
т.е. по переменной $y$:

$$
	\overline G(p,q,\mu)=
	p\min{\{\frac{q}{\mu};
	\frac{1-q}{2(1-\mu)}\}}
	+(1-p)\min\{\frac{q}{2\mu};\frac{1-q}{1-\mu}\},
$$

а игрок \textbf{П} осредняет свёртку критерия по стратегиям игрока \textbf{С},
т.е. по переменной $x$:

$$
	\overline L(p, q, \lambda) =
	\frac{1}{2}\big \{q(3\lambda+p-2)+(2-p)(1-\lambda)\big \}.
$$

Мы установили функции выигрыша игроков. Теперь задачу можно формализовать и представить
как семейство игр двух игроков в нормальный форме:

$$
	\bigg \langle 
		\{\textrm{\textbf{С}, \textbf{П}}\},		
		\{Q, \: P\},		
		\{\overline L(p,q,\lambda), - \overline G(p,q,\mu) \}
	\bigg \rangle 
	, \; (\lambda, \mu) \in \Lambda \times  M
$$

Для нас представляет интерес множество оптимальных точек \eqref{def:optimal_strategy}.