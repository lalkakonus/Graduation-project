\subsection{Значение игры для игрока Преподаватель}

Вернёмся к рассмотрению случая Т=1.
Вычислим математическое ожидание от имеющегося векторного критерия 
\eqref{eq:player_criterion}
, где $x$ и $y$ возьмём как случайные величины с распределениями 
\eqref{eq:probability_1}
и соответствующими обозначениями величин $p$ и $q$. В таком случае имеем:

$$
	\mathbb{E}_{xy} [F(x,y)]=
	\Big \langle
		\dfrac{q(1+p)}{2};
		\dfrac{(1-q)(2-p)}{2}
	\Big\rangle
$$

В утверждении 1 мы установили, что любая пара
$(p^0, q^0) \in [0, 1]^{2}$ является оптимальной. 
Рассмотрим это как множество точек на плоскости $X,Y$ зависящие от двух
параметров $(p,q)\in[0,1]^2$
$$
	\begin{cases}
		x = \dfrac{q(1 + p)}{2} \\
		y = \dfrac{(1 - q)(2 - p)}{2}  
	\end{cases}
	\Rightarrow
	\begin{cases}
		q = \dfrac{2x}{1 + p} \\
		y = \dfrac{(1 - \dfrac{2x}{1 + p})(2 - p)}{2}		
	\end{cases}
	\Rightarrow
	\begin{cases}
		q = \dfrac{2x}{1 + p} \\
		y = \dfrac{(1 + p - 2x)(2 - p)}{2(1 + p)}
	\end{cases}
$$
Найдём максимальные значения, которые может принимать $y(x, p)$ при фиксированном $x$:
$$
	\dfrac{\partial y(x, p)}{\partial p}=\frac{3x}{(p+1)^2} - \frac{1}{2} = 0 
	\Rightarrow
	p = \sqrt{6x} - 1.
$$
Введём обозначение $p_0 = \sqrt{6x} - 1$. Поскольку область определения 
$p_0 \in [0, 1]$, то $p_0 = 1 \Rightarrow x = \dfrac{2}{3}$ и 
$p_0 = 0 \Rightarrow x = \dfrac{1}{6}$.
$$
	y_{max}(x) = 
	\begin{cases}
		\max \{ y(x, 0), \: y(x, 1), \: y(x, p_0) \}, & 
		x \in \big[ \frac{1}{6}, \frac{2}{3} \big]
		\\
		\max \{ y(x, 0), \: y(x, 1) \}, &
		x \in \big[0, \frac{1}{6} \big] \cap \big[\frac{2}{3}, 1\big]
	\end{cases}
$$
учитывая, что
$
	y(x, 0) = 1 - 2x, \;
	y(x, 1) = \dfrac{1 - x}{2}, \;  	
	y(x, p_0) = \dfrac{(\sqrt{6x} - 2x)(3 - \sqrt{6x})}{2 \sqrt{6x}}
$
получим уравнения верхней и нижней огибающей области на графике (рис. 
\ref{fig:EF_1}).

$
	y_{min}(x) =
	\begin{cases}
		\dfrac{1 - x}{2}, & x \in \big[0, \frac{1}{3} \big] 
		\\
		1 - 2x, & x \in \big( \frac{1}{3}, 1\big]
	\end{cases}
	\qquad
	y_{max}(x) =
	\begin{cases}
		1 - 2x, & x \in \big[0, \frac{1}{6} \big] 
		\\
		\frac{(\sqrt{6x} - 2x)(3 - \sqrt{6x})}{2 \sqrt{6x}}, &
		x \in \big(\frac{1}{6}, \frac{2}{3} \big]
		\\
		\dfrac{1 - x}{2}, & x \in \big(\frac{2}{3}, 1 \big]
	\end{cases}
$

Теперь для всех оптимальных стратегий, т.е. пар 
$(p^0, q^0) \in [0, 1]^{2}$ изобразим на графике
(рис. \ref{fig:EF_1}) значения математическое ожидание векторного критерия 
в этой точке. Аналогичными рассуждениями для случая Т=2 получаем область на
рис. \ref{fig:EF_2}. Таким образом мы спроецировали множества в
критериальном пространстве представляющие собой максимум среднего вектора
критериев:

\begin{figure}[H]
	\centering
	\begin{subfigure}[b]{0.47 \textwidth}
		\centering		
		\includegraphics[width=\textwidth]{part_3/Figure_8}
		\caption{}
		\label{fig:EF_1}
	\end{subfigure}	
	\begin{subfigure}[b]{0.47 \textwidth}
		\centering		
		\includegraphics[width=\textwidth]{part_3/Figure_5}
		\caption{}
		\label{fig:EF_2}
	\end{subfigure}
	\caption{}	
\end{figure}

Видно, что при увеличении степени дискретизации, графики
приближаются к той форме, которую имеют соответствующие множества
в непрерывной задаче \cite{novikova}.