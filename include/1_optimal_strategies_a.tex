\section{Решения игры. \\Параметр дискретизации Т=1}

Для поиска оптимальных стратегий сначала необходимо найти точки максимума и минимума 
функций выигрыша \textbf{С} и \textbf{П} соответственно:
$$
	q^*(p, \mu) = \arg \max \limits_{q \in Q_1} \overline G(p, q, \mu)
	\textrm{ и }
	p^*(q, \lambda) = \arg \min \limits_{p \in P} \overline L(p, q, \lambda).
$$
Из статьи \cite{novikova} следует, что:

\begin{equation}
	p^*(q, \lambda)=
	\arg \min \limits_{p \in P} \overline L(p, q, \lambda) =
	\begin{cases}
		0, & q > 1 - \lambda \\
		1, & q < 1 - \lambda \\
		[0,1], & q = 1 - \lambda
	\end{cases}
	\label{eq:argmin_L_1}
\end{equation}
Аналогичными рассуждениями получаем, что:
\begin{equation}
	q^*(p, \mu) = \arg \max \limits_{q \in Q} \overline G(p, q, \mu) =
	\begin{cases}
		\dfrac{\mu}{2 - \mu}, & p \geqslant 1 - \mu \\
		\dfrac{2\mu}{1 + \mu}, & p \leqslant 1 -\mu \\
	\end{cases}
	\label{eq:argmax_G_1}
\end{equation}

Докажем утверждение характеризующее множество оптимальных пар данной игры.

\newtheorem{State}{Утверждение}\label{State:opt_strat_1}
\begin{State}
	Любая пара $(p^*, q^*) \in [0, 1]^2$ является оптимальной, т.е.  
	$\forall \; (p^*, q^*) \in [0, 1]^2 \; \exists \; 
	(\mu, \lambda) \in M \times \Lambda$
	такие, что верно \eqref{def:optimal_strategy}.
\end{State}


\textbf{Доказательство}

Зафиксируем некоторую пару  $(p^*, q^*) \in [0, 1]^2$ и найдём такие 
$\hat \mu (p^*, q^*) \in M$ и $\hat \lambda (p^*, q^*) \in \Lambda$ 
что верно \eqref{def:optimal_strategy} т.е. что пара стратегий является оптимальной.

\textbf{(1)} Определим $\hat \lambda(p^*, q^*)$ и покажем что 
$p^* \in \arg \min \limits_{p \in P} \overline L(p, q^*, \hat \lambda)$.
Возьмём $\hat \lambda := 1 - q^*$ тогда поскольку 
$\arg \min\limits_{p \in P} \overline L(p, q^*, \hat \lambda) = [0,1]$ 
при $ q^* = 1 - \hat \lambda$, 
то $p^* \in \arg \min \limits_{p \in P} \overline L(p, q^*, \hat \lambda)$.

\textbf{(2)} Определим $\hat \mu(p^*, q^*)$ и покажем что 
$q^* \in \arg \max \limits_{q \in Q} \overline G(p^*, q, \hat \mu) $.
По имеющемуся $q^*$ решим уравнения $\dfrac{\mu}{2 - \mu}=q^*$ и
$\dfrac{2 \mu}{1 + \mu}=q^*$, относительно переменной $\mu$:

$$
	q^*=\dfrac{2 \mu}{1 + \mu} \Rightarrow \mu = \frac{q^*}{2 - q^*} \qquad \qquad
	q^*=\dfrac{\mu}{2 - \mu} \Rightarrow \mu = \frac{2 q^*}{1 + q^*}
$$

Введём обозначения $\mu_1(q) = \dfrac{q}{2 - q}$ и 
$\mu_2(q) = \dfrac{2 q}{1 + q}$. Заметим, что при $q^* \in [0, 1]$ 
верно $0 \leqslant \mu_2(q^*) \leqslant \mu_1(q^*) \leqslant 1$.
В таком случае поскольку $1 - p^* \in [0, 1]$, то при фиксированных переменных 
$(p^*, q^*)$ будет реализован один и только один из 3-х вариантов:

\textbf{(a)}
$1 - p^* \leqslant \mu_2(q^*)$, т.е.  $1 - p^* \leqslant \dfrac{q^*}{2 - q^*}$.
Возьмём $\hat \mu := \mu_1 = \dfrac{2 q^*}{1 + q^*}$.

Действительно, при таком выборе $\hat{\mu}$ имеем: 
\begin{gather*}
	1 - p^* \leqslant \mu_2 \leqslant \mu_1 = \hat \mu \Rightarrow
	p^* \geqslant 1 - \hat \mu \Rightarrow 
	\\	
	\Rightarrow \arg \max \limits_{q \in Q} \overline G(p^*, q, \hat \mu) =
	\dfrac{\hat{\mu}}{2-\hat{\mu}} =
	\dfrac{\dfrac{2 q^*}{1 + q^*}}{2 - \dfrac{2 q^*}{1 + q^*}} =
	q^*
\end{gather*} 


\textbf{(b)}
$\mu_2(q^*) < 1-p^* \leqslant \mu_1(q^*)$, т.е.  
$\dfrac{q^*}{2 - q^{*}} < 1 - p^* \leqslant \dfrac{2 q^*}{1 + q^*}$.
Возьмём $\hat \mu := \mu_1 = \dfrac{2 q^*}{1 + q^*}$.
Действительно, при таком выборе $\hat \mu$ имеем:

\begin{gather*}
	1 - p^* \leq \mu_1 = \hat \mu 
	\; \Rightarrow \;
	p^* \geqslant 1 - \hat \mu \Rightarrow 
	\\
	\Rightarrow  \arg \max \limits_{q \in Q} \overline G(p^*, q, \hat \mu) =
	\dfrac{\hat \mu}{2 - \hat \mu} =
	\dfrac{\dfrac{2 q^*}{1 + q^*}}{2 - \dfrac{2 q^*}{1 + q^*}}=
	q^* \Rightarrow
\end{gather*} 



\textbf{(c)} 
$\mu_1(q^*) < 1 - p^*$ т.е. $\dfrac{2 q^*}{1 + q^*} < 1 - p^*$.
Возьмём $\hat \mu := \mu_2 = \dfrac{q^*}{2 - q^*}$
Действительно, при таком выборе $\hat \mu$ имеем:

\begin{gather*}
	1 - p^* > \mu_1 \geqslant \mu_2 = \hat \mu \Rightarrow
	p^* \leqslant 1 - \hat \mu \Rightarrow 
	\\	
	\Rightarrow \arg \max \limits_{q \in Q} \overline G(p^*, q, \hat \mu) =
	\dfrac{2 \hat \mu}{1 + \hat \mu} =
	\dfrac{2 \dfrac{q^*}{2 - q^*}}{1 + \dfrac{q^*}{2 - q^*}} =
	q^*.
\end{gather*}

Теперь для любой точки $(p^*,q^*) \in [0,1]^2$ можем указать
$(\hat \mu(p^*,q^*), \hat \lambda(p^*,q^*))$ такие, что верно \eqref{def:optimal_strategy},
а именно:
 
$$
(\hat \mu(p^*,q^*), \hat \lambda(p^*,q^*)) = 
\begin{cases}
	\Big(\dfrac{q^*}{2 - q^*}, \: 1 - q^* \Big), &
	\dfrac{2 q^*}{1 + q^*} < 1 - p^*
	\\
	\Big(\dfrac{2 q^*}{1 + q^*}, \: 1 - q^* \Big), &
	1 - p^* \leqslant \dfrac{q^*}{2 - q^*}
	\\
	\Big(\dfrac{2 q^*}{1 + q^*}, \: 1 - q^*\Big), & 	
	\dfrac{q^*}{2 - q^{*}} < 1 - p^* \leqslant \dfrac{2 q^*}{1 + q^*}
\end{cases}
$$

\textbf{Утверждение доказано.}