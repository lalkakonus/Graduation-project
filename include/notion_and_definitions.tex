\section{Основные понятия и определения}

Определим формально модель игры с несколькими участниками в общем виде.
Есть конечное \textit{множество игроков} $A = \{1, 2, ..., m \}$. 
Каждый игрок $a \in A$ имеет 
\textit{множество чистых стратегий} $S_a=\{1,2,...,n_a\}$, при этом 
\textit{игровой ситуацией} или просто \textit{ситуацией}
называется $m$-мерный вектор:
\begin{equation}
	\label{eq:game_situation}
	\textbf{s} = (s_1, \, s_2, \ldots , \, s_m) \in
	S_1 \times S_2 \times \ldots \times S_m,
\end{equation}
где $X \times Y$ обозначает декартово произведение множеств $X$ и $Y$.
\textit{Функция выигрыша} $F_a, \: a \in A$ обозначает выигрыш игрока при
конкретной ситуации в игре. Она определена для каждого игрока и имеет 
вид:
\begin{equation}
	\label{eq:single_dim_payoff_function}
	F_a: S_1 \times S_2 \times \ldots \times S_m \rightarrow 
	\mathbb R, \; a \in A
\end{equation}

\newtheorem{Defenition}{Определение}
\begin{Defenition}
	Игрой в нормальной форме называется совокупность:
	\begin{equation}
		G = \big \langle A, S, F \big \rangle
		\label{eq:normal_form_game}
	\end{equation}
	где: 
 
	$ A = \{1, \, 2, ..., \, m\}$ - множество игроков,

	$ S = \{S_1, \, S_2, ..., \, S_m\}$ - множество наборов чистых
	стратегий игроков,

	$ F = \{F_1, \, F_2, ..., \, F_m\}$ - множество функций выигрыша
	игроков. 
	\cite{vasin}
\end{Defenition}

\begin{flushleft}
Теперь введём фундаментальное понятие в теории игр --
\textit{равновесие по Нэшу}:
\end{flushleft}
\begin{Defenition}
	Ситуация $\textbf{s} = (s_1^0, \, s_2^0, ..., \, s_m^0)$ называется
	\cite{vasin} равновесием по Нэшу в игре \eqref{eq:normal_form_game},
	если:
	\begin{equation}
		\max \limits_{s_a \in S_a} 
		F_a(s_1^0, ..., s_{a-1}^0, s_a, s_{a+1}^0..., s_m^0)=
		F_a(s_1^0, ..., s_{a-1}^0, s_a^0, s_{a+1}^0..., s_m^0),
		\: \forall \; a \in A.
		\label{eq:nash_equilibrium}
	\end{equation}
\end{Defenition}

Смысл этого определения заключается в том, что 
при ситуации в игре, которая является равновесием по Нэшу,
ни одному игроку индивидуально не выгодно отклоняться от своей текущей 
стратегии.

До этого мы рассматривали функции выигрыша 
игроков, которые имели вид
\eqref{eq:single_dim_payoff_function}, т.е. каждому игроку 
соответствовало одно значение, зависящее от ситуации игры.
Однако, не всегда интересы могут быть выражены одним критерием. Часто 
возникают разные оценки качества принимаемого решения, причем они могут 
быть противоречивыми и их нельзя свести друг у другу. Например 
характеристиками решения могут быть такие значения 
как \textit{(время, деньги)} или 
\textit{(математическое ожидание, дисперсия)}. Следуя этим рассуждениям  
рассмотрим такое обобщение игры \eqref{eq:normal_form_game}, что функция
выигрыша игроков имеет вид:
\begin{equation}
	F: S_1 \times S_2 \times ... \times S_m \rightarrow \mathbb R^n
	, \; n \in \mathbb{N}
	\label{eq:multidem_payoff_function}
\end{equation}

Такое обобщение ближе к реальным ситуациям
в которых критерием выбора представляет собой несколько значимых параметров.
Для примера такой игры можно привести задачу выбора машины: 
допустим покупателю важно чтобы машина имела большую мощность, 
достаточный уровень безопасности 
и мало стоила, продавцу же важно, чтобы она стоила как можно дороже и 
кроме того следует продавать те машины, которые плохо покупают. 
Таким образом мы получили игру, в которой игроки имеют три и два критерия
соответственно, которые важны для них при выборе стратегии. 
Пока что мы допустили существование игры с функцией выигрыша
вида \eqref{eq:multidem_payoff_function}. Формализация и подробное 
описание будет позже.

Приведённые выше обобщения приводят нас к другому разделу
математики, а именно -- \textit{многокритериальной оптимизации}.
Рассмотрим следующую задачу которая относится к этой области.
\begin{equation}
	\label{eq:multi_criteral_problem}
	\Max \limits_{x \in X} F(x), \; F(x)=({f}_1(x),\ldots, {f}_m(x))
	, \; X \subseteq \mathbb{R}^n,
\end{equation}
где $f_i: \mathbb{R}^n \rightarrow \mathbb{R}, \; i = 1, \ldots, m.$
Это задача заключается в том, что у нас есть $n$-мерная функция,
которая представляет собой множество значений критериев, зависящих
от параметров, которые принадлежат некоторому множеству.
Особенность заключается в том, что правило сравнения двух векторов
не определено однозначно, т.е. в общей задаче не всегда можно точно сказать,
какой из двух векторов значений функции предпочтительнее.
В примере про покупку автомобиля можно сказать, что если 
одна машина более мощная и менее дорогая чем вторая, то она предпочтительнее,
однако, если она более мощная, но при этом и стоит дороже, то
правило их сравнения не определено. 
Введём понятие \textit{оптимальных (эффективных) по Парето} и 
\textit{оптимальных (эффективных) по Слейтеру}
векторов в задаче \eqref{eq:multi_criteral_problem}.
И значением максимума в задаче \eqref{eq:multi_criteral_problem},
который в отличие от скалярного максимума записывается с заглавной буквы,
будем считать \textit{множество Слейтера} в критериальном пространстве 
$\mathbb{R}^m$.

\begin{Defenition}
	Допустимое решение $\hat{x}\in{X}$ называется 
	оптимальным по Слейтеру (оптимальным по Парето) для задачи		
	\eqref{eq:multi_criteral_problem},
	если \textbf{не} существует $x\in{X}$ такого, что 
	$f_k(x)>f_k(\hat{x}) \; (f_k(x) \geqslant f_k(\hat{x}))$ для всех
	$k = 1, \ldots ,m$. Множество всех оптимальных по Слейтеру 
	(оптимальных по Парето) решений задачи 
	\eqref{eq:multi_criteral_problem} называется 
	\textit{множеством Слейтера} (\textit{множеством Парето}) 
	задачи \eqref{eq:multi_criteral_problem}.
	\cite{ehrgott}
\end{Defenition}

Другими словами это такое множество значений, 
при котором значение каждого частного показателя, характеризующего систему, 
не может быть улучшено без ухудшения других.

Для поиска множества Слейтера существуют разные методы
\cite{ehrgott}, \cite{blackwell},
в текущей работе исследуется \textit{метод свёрток} \cite{germeyer}. 
Метод заключается в том, что задача \eqref{eq:multi_criteral_problem}
заменяется параметрическим семейством скалярных задач:
$$
	\max\limits_{x \in X} C(\{f_i\}_{i=1}^{m}, \lambda, x), \;
	\lambda \in \Lambda
$$
где: 
 
$C$ – функция свертки частных критериев $\{f_i\}_{i=1}^m$ задачи 
\eqref{eq:multi_criteral_problem} в единый скалярный критерий,
 
$\lambda$ – параметр свертки заданный на некоторой
области определения $\Lambda$.
\newline

В текущей работе рассмотрены две различные свёртки --
\textit{линейная свёртка} и \textit{обратная логическая свёртка}:

\begin{Defenition}
	Линейной свёрткой с параметром $\lambda$ для функции критериев задачи
	\eqref{eq:multi_criteral_problem} называется 	\cite{ehrgott}
	функция:
	\begin{equation}
		\label{eq:linear_scalarization}
		L(\{f_i\}_{i=1}^{m}, \lambda, x) = 
		\sum_{i=1}^{n} \lambda_i f_i,
	\end{equation}
	где  
	\begin{equation}
		\label{eq:Lambda}
		\lambda \in 
		\Lambda = \{
			(\lambda_1, \ldots, \lambda_m) \:
			| \: \sum_{i=1}^m \lambda_i = 1, \: 
			  \lambda_i \geq 0 \; i = 1, \ldots m 
		\}.		
	\end{equation}
\end{Defenition}

\begin{Defenition}
	Свёрткой Гермейера с параметром $\mu$ для 
	функции критериев задачи \eqref{eq:multi_criteral_problem}
	называется \cite{germeyer} функция:
	\begin{equation}
		\label{eq:germeyer_scalarization}	
		G(\{f_i\}_{i=1}^{m}, \mu, x)=
		\min \limits_{i: \mu_i > 0} \mu_i f_i,
	\end{equation}
	где 
	\begin{equation}
		\label{eq:Mu}	
		\mu \in 
		M = \{
			(\mu_1, \ldots, \mu_m) \:
			| \: \sum_{i=1}^m \mu_i = 1, \: 
			  \mu_i \geq 0 \; i = 1, \ldots m 
		\}.
	\end{equation}
\end{Defenition}

Применение линейной свёртки в задачах вида
\eqref{eq:multi_criteral_problem} обосновывается теоремой Карлина:

\newtheorem{Theorem}{Теорема}
\begin{Theorem}[С. Карлин]
	Рассмотрим задачу \eqref{eq:multi_criteral_problem}. 
	Пусть множество $X \subseteq \mathbb{R}^n$ выпукло,
	а функции $f_1, \ldots, f_m$ - вогнуты на нём.
	Если $x^*$ – эффективная по Парето точка,
    тогда существует вектор $\lambda \in \Lambda$ из 
    \eqref{eq:Lambda} такой, что $x^*$ является точкой 
    максимума функции \eqref{eq:linear_scalarization} по переменной
    $x$. \cite{carlin}
\end{Theorem}

Ю.Б.Гермейером была предложена свертка, которая также аппроксимирует 
множество Слейтера. Её применение в многокритериальных 
задачах обосновывается следующей теоремой:

\begin{Theorem}[Ю.Б. Гермейер]
	Рассмотрим задачу \eqref{eq:multi_criteral_problem}.     
    Пусть $x^*$ - эффективная по Слейтеру точка, 
    причем $f_1(x^*)>0, \ldots, f_m(x^*)>0$.
    Тогда существует вектор $\mu \in M$ из \eqref{eq:Mu} 
    и $x^*$ является точкой максимума функции 
    \eqref{eq:germeyer_scalarization} по переменной $x$. 
    \cite{germeyer} 
\end{Theorem}

В работе будет использоваться модификация свёртки Гермейера -
\textit{обратная логическая свертка}. Она отличается
только тем, что веса (параметры свёртки) стоят в знаменателе, 
а не в числителе.

\begin{Defenition}
	Обратной логической свёрткой с параметром $\mu$ для 
	функции критериев задачи \eqref{eq:multi_criteral_problem}
	называется функция:
	
	\begin{equation}
		G(\{f_i\}_{i=1}^{m}, \mu, x)=
		\min \limits_{i: \mu_i > 0} \frac{f_i}{\mu_i},
		\label{eq:RL_scalarization}	
	\end{equation}
	где $\mu \in M$ определено в \eqref{eq:Mu}.
\end{Defenition}

Вернёмся к рассмотрению некооперативных игр.
Если множество чистых стратегий у игроков конечно, то игра
называется \textit{конечная}. Для конечной игры
определим обобщение модели, 
а именно - \textit{смешанное расширение игры}.
Смысл смешанного расширения игры заключается в том, что
каждый игрок выбирает любую из своих чистых стратегий с некоторой
фиксированной вероятностью, и его стратегией является
не одна чистая стратегия, а вероятностное  распределение над множеством 
его чистых стратегий. Выигрышем игрока в таком случае считаем взвешенный
выигрыш по всем ситуациям с весами соответствующими вероятностям данной
ситуации. Определим эти понятия формально.

Пусть в игре 
$G = \big \langle A, S, F \big \rangle$. $A$ -- конечное множество игроков,
$A = \{1, 2, ..., m\}$, причём
множества чистых стратегии $S_a=\{1, 2,...,n_a\} \; a \in A$ конечны.  
\textit{Смешанной стратегией} игрока $a \in A$ называется 
вероятностное распределение над множеством чистых стратегий 
$S_a$ игрока $a \in A$:
$$
	\pi^a \in P_a
	= \{ 
		(\pi^a_1, \pi^a_2, ..., \pi^a_{n_a}) \in [0,1]^{n_a}\: | \:
		\sum \limits_{i=1}^{n_a} \pi^a_i = 1
	\}
$$
где $\pi^a_i$ - это вероятность выбора игроком $a \in A$ чистой 
стратегии $s^a_i$ в качестве реальной стратегии игрока,
а $[0,1]^n$ обозначает $n$-ую декартову степень отрезка $[0,1]$.
Симплекс $P_a$ называется \textit{множеством смешанных стратегий игрока}.
Введём обозначение для заданного набора стратегий:
$$
	\pi = (\pi^1, \ldots, \pi^m) \in P = P_1 \times \ldots \times P_m,
$$
и вероятности реализации ситуации \textbf{s} из 
\eqref{eq:game_situation}:
$$
	p(s|\pi) = \prod_{a \in A} \pi^a_{s_a}.
$$
Тогда математическое ожидание выигрыша игрока $a \in A$
задаётся функцией: 
$$
	\overline F_a(\pi) = \sum \limits_{s \in S}p(s|\pi)F_a(s),
$$
где функция $F_a$ - определена в \eqref{eq:normal_form_game}.
Таким образом смешанное расширение игры в нормальной форме определяется
следующим образом:

\begin{Defenition}
	Смешанным расширением игры в нормальной форме называется 
	совокупность:
	\begin{equation}
		\overline G = \lbrace A, P, \overline F \rbrace	
		\label{eq:ext_normal_form_game}
	\end{equation}
	где: 
 
	$A = \{1, 2, ..., m\}$ -- множество игроков,

	$P = P_1 \times \ldots \times P_m$ -- множество наборов смешанных 
	стратегий игроков,

	$\overline F = \{
		\overline F_1, \overline F_2, ..., \overline F_m
	\}$
	-- множество функций выигрыша игроков.
\end{Defenition}

Ситуации равновесия \eqref{eq:nash_equilibrium} игры $\overline G$
будем называть \textit{ситуациями равновесия в смешанных стратегиях игры $G$}
или \textit{смешанными равновесиями по Нэшу}. 

Теперь в игре \eqref{eq:ext_normal_form_game} будем считать, что функции
выигрыша $F_a, \; a \in A$ имеют вид \eqref{eq:multidem_payoff_function}.
Обозначим через $S_a(\pi || a), \; a \in A$ множество Слейтера задачи  
$
	\Max \limits_{\pi \: : \: \pi^a \in P^a} \overline F_a(\pi), 
$
в которой все координаты $\pi$, кроме $a$ фиксированы.

\begin{Defenition}
	Решением игры \eqref{eq:ext_normal_form_game} согласно \cite{blackwell} 
	является множество ситуаций $P^*$:
	\begin{equation}
		P^* = \{ 
			\pi = (\pi^1, \ldots, \pi^m) \in P \: | \:
			\pi^a \in S_a(\pi || a), \; a \in A
		\}
		\label{eq:game_solution}
	\end{equation}
\end{Defenition}

В случае конечных многокритериальных игр Шепли свел \cite{shapley}
описание данного множества к семейству задач поиска значений 
скалярных игр с 
функциями выигрышей -- ЛС частных критериев при произвольном наборе
весовых коэффициентов, своих у каждого игрока.
В случае скалярной игры осреднение однозначно, а  для скаляризованной
вектор-функции могут быть разные варианты.

Целью данной выпускной квалификационной работы является исследование 
применения разных сверток в игре с векторным выигрышем.