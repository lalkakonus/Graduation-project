\begin{flushleft}


Для введём обозначения для вероятностей:
\[
\begin{cases}
P(X=0)=1-q \\
P(X=1)=q \\
\end{cases}
\textrm{и }
\begin{cases}
P(Y=1)=1-p \\
P(Y=2)=p \\
\end{cases}
\]
И укажем области определения переменных, используемых ниже:
\[
p \in [0, 1],\quad q \in [0, 1],\quad
\mu \in [0, 1],\quad \lambda \in [0, 1]
\]

И предположим, что Студент осредняет свой критерий по переменной $x$ и получает:
$$F(q, y)=(f_1(q, y), f_2(q, y)) =\big(\frac{q y}{2}, \frac{1-q}{y}\big),$$
а Преподователь осредняет свой критерий по переменной $y$ и получает
$$F(x, p)=(f_1(x, p), f_2(x, p)) =\big(\frac{(1+p)\sqrt{x}}{2}, \frac{(2-p)\sqrt{1-x}}{2}\big)$$

\vspace{5mm}

Рассмотрим вариант, когда Студент использует обратную логическую свёртку и осредняет её по переменной $y$:
\begin{equation} 
M(p,q,\mu)=p\min{\{\frac{q}{\mu};\frac{1-q}{2(1-\mu)}\}}+(1-p)\min\{\frac{q}{2\mu};\frac{1-q}{1-\mu}\},
\end{equation}

a Преподаватель использует линейную свёртку и осредняет её по переменной $x$:
\begin{equation} 
L(p,q,\lambda)=\frac{1}{2}\big \{q(3\lambda+p-2)+(2-p)(1-\lambda)\big \}.
\end{equation} 

Игроки пытаются максимизировать или минимизировать сответствующие функции с помощью выбора своей стратегии
\[
\begin{cases} M(p,q,\mu)\rightarrow\max\limits_{q} \\
L(p,q,\lambda)\rightarrow\min\limits_{p}\end{cases}
\]

\vspace{5mm}

Для исследования оптимальных стратегий необходимо установить значения $p$ и $q$, при которых
эти функции достигают минимума и максимума соответственно. Этот вопрос уже был исследован. Из статьи [3] следует, что:
\begin{equation}
p^{*}(q, \lambda)=\argmin\limits_{p}L(p,q,\lambda)=\begin{cases}0,& q>1-\lambda \\
1,& q<1-\lambda \\
\forall,& q=1-\lambda\end{cases}
\end{equation}

Из курсовой [2] следует, что
\begin{equation}
q^{*}(p, \mu)=\argmax\limits_{q}M(p,q,\mu)=
\begin{cases}
\frac{\mu}{2-\mu},& p+\mu-1\geq 0 \\
\frac{2\mu}{1+\mu},& p+\mu-1\leq 0 \\
\end{cases}
\end{equation}

\end{flushleft}